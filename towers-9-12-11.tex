\documentclass[epsfig,12pt]{article}
\usepackage{epsfig}
\usepackage{graphicx}
\usepackage{rotating}

%%%%%%%%%
\usepackage{latexsym}
\usepackage{amsmath}
\usepackage{amssymb}
\usepackage{relsize}
\usepackage{geometry}
\geometry{letterpaper}
\usepackage{color}
\usepackage{bm}
%\usepackage{showlabels}
%%%%%%%%%%%%

\def\beq{\begin{equation}}
\def\eeq{\end{equation}}
\def\beqn{\begin{eqnarray}}
\def\eeqn{\end{eqnarray}}
\def\Tr{{\rm Tr}}
\newcommand{\nfour}{${\cal N}=4\;$}
\newcommand{\ntwo}{${\mathcal N}=2\;$}
\newcommand{\none}{${\mathcal N}=1\,$}
\newcommand{\ntt}{${\mathcal N}=(2,2)\,$}
\newcommand{\nzt}{${\mathcal N}=(0,2)\,$}
\newcommand{\cpn}{CP$(N-1)\,$}
\newcommand{\ca}{{\mathcal A}}
\newcommand{\cell}{{\mathcal L}}
\newcommand{\cw}{{\mathcal W}}
\newcommand{\cs}{{\mathcal S}}
\newcommand{\vp}{\varphi}
\newcommand{\pt}{\partial}
\newcommand{\ve}{\varepsilon}
\newcommand{\gs}{g^{2}}
\newcommand{\zn}{$Z_N$}
\newcommand{\cd}{${\mathcal D}$}
\newcommand{\cde}{{\mathcal D}}
\newcommand{\cf}{${\mathcal F}$}
\newcommand{\cfe}{{\mathcal F}}

\newcommand{\gsim}{\lower.7ex\hbox{$
\;\stackrel{\textstyle>}{\sim}\;$}}
\newcommand{\lsim}{\lower.7ex\hbox{$
\;\stackrel{\textstyle<}{\sim}\;$}}

\renewcommand{\theequation}{\thesection.\arabic{equation}}

%%%%%%%%%%%%%
%%%%%%%%%%%
%% common definitions
\def\stackunder#1#2{\mathrel{\mathop{#2}\limits_{#1}}}
\def\beqn{\begin{eqnarray}}
\def\eeqn{\end{eqnarray}}
\def\nn{\nonumber}
\def\baselinestretch{1.1}
\def\beq{\begin{equation}}
\def\eeq{\end{equation}}
\def\ba{\beq\new\begin{array}{c}}
\def\ea{\end{array}\eeq}
\def\be{\ba}
\def\ee{\ea}
\def\stackreb#1#2{\mathrel{\mathop{#2}\limits_{#1}}}
\def\Tr{{\rm Tr}}
%\newcommand{\gsim}{\lower.7ex\hbox{$\;\stackrel{\textstyle>}{\sim}\;$}}
% \newcommand{\lsim}{\lower.7ex\hbox{$
%\;\stackrel{\textstyle<}{\sim}\;$}}
%\newcommand{\nfour}{${\mathcal N}=4$ }
%\newcommand{\ntwo}{${\mathcal N}=2$ }
\newcommand{\ntwon}{${\mathcal N}=2$}
\newcommand{\ntwot}{${\mathcal N}= \left(2,2\right) $ }
\newcommand{\ntwoo}{${\mathcal N}= \left(0,2\right) $ }
%\newcommand{\none}{${\mathcal N}=1$ }
\newcommand{\nonen}{${\mathcal N}=1$}
%\newcommand{\vp}{\varphi}
%\newcommand{\pt}{\partial}
%\newcommand{\ve}{\varepsilon}
%\newcommand{\gs}{g^{2}}
%\newcommand{\qt}{\tilde q}
\renewcommand{\theequation}{\thesection.\arabic{equation}}

%%
\newcommand{\p}{\partial}
\newcommand{\wt}{\widetilde}
\newcommand{\ov}{\overline}
\newcommand{\mc}[1]{\mathcal{#1}}
\newcommand{\md}{\mathcal{D}}

\newcommand{\GeV}{{\rm GeV}}
\newcommand{\eV}{{\rm eV}}
\newcommand{\Heff}{{\mathcal{H}_{\rm eff}}}
\newcommand{\Leff}{{\mathcal{L}_{\rm eff}}}
\newcommand{\el}{{\rm EM}}
\newcommand{\uflavor}{\mathbf{1}_{\rm flavor}}
\newcommand{\lgr}{\left\lgroup}
\newcommand{\rgr}{\right\rgroup}

\newcommand{\Mpl}{M_{\rm Pl}}
\newcommand{\suc}{{{\rm SU}_{\rm C}(3)}}
\newcommand{\sul}{{{\rm SU}_{\rm L}(2)}}
\newcommand{\sutw}{{\rm SU}(2)}
\newcommand{\suth}{{\rm SU}(3)}
\newcommand{\ue}{{\rm U}(1)}
%%%%%%%%%%%%%%%%%%%%%%%%%%%%%%%%%%%%%%%
%  Slash character...
\def\slashed#1{\setbox0=\hbox{$#1$}             % set a box for #1
   \dimen0=\wd0                                 % and get its size
   \setbox1=\hbox{/} \dimen1=\wd1               % get size of /
   \ifdim\dimen0>\dimen1                        % #1 is bigger
      \rlap{\hbox to \dimen0{\hfil/\hfil}}      % so center / in box
      #1                                        % and print #1
   \else                                        % / is bigger
      \rlap{\hbox to \dimen1{\hfil$#1$\hfil}}   % so center #1
      /                                         % and print /
   \fi}                                        %

%%EXAMPLE:  $\slashed{E}$ or $\slashed{E}_{t}$

%%

\newcommand{\LN}{\Lambda_\text{SU($N$)}}
\newcommand{\sunu}{{\rm SU($N$) $\times$ U(1) }}
\newcommand{\sunun}{{\rm SU($N$) $\times$ U(1)}}
\def\cfl {$\text{SU($N$)}_{\rm C+F}$ }
\def\cfln {$\text{SU($N$)}_{\rm C+F}$}
\newcommand{\mUp}{m_{\rm U(1)}^{+}}
\newcommand{\mUm}{m_{\rm U(1)}^{-}}
\newcommand{\mNp}{m_\text{SU($N$)}^{+}}
\newcommand{\mNm}{m_\text{SU($N$)}^{-}}
\newcommand{\AU}{\mc{A}^{\rm U(1)}}
\newcommand{\AN}{\mc{A}^\text{SU($N$)}}
\newcommand{\aU}{a^{\rm U(1)}}
\newcommand{\aN}{a^\text{SU($N$)}}
\newcommand{\baU}{\ov{a}{}^{\rm U(1)}}
\newcommand{\baN}{\ov{a}{}^\text{SU($N$)}}
\newcommand{\lU}{\lambda^{\rm U(1)}}
\newcommand{\lN}{\lambda^\text{SU($N$)}}
%\newcommand{\Tr}{{\rm Tr\,}}
\newcommand{\bxir}{\ov{\xi}{}_R}
\newcommand{\bxil}{\ov{\xi}{}_L}
\newcommand{\xir}{\xi_R}
\newcommand{\xil}{\xi_L}
\newcommand{\bzl}{\ov{\zeta}{}_L}
\newcommand{\bzr}{\ov{\zeta}{}_R}
\newcommand{\zr}{\zeta_R}
\newcommand{\zl}{\zeta_L}
\newcommand{\nbar}{\ov{n}}

\newcommand{\CPC}{CP($N-1$)$\times$C }
\newcommand{\CPCn}{CP($N-1$)$\times$C}

\newcommand{\lar}{\lambda_R}
\newcommand{\lal}{\lambda_L}
\newcommand{\larl}{\lambda_{R,L}}
\newcommand{\lalr}{\lambda_{L,R}}
\newcommand{\bla}{\ov{\lambda}}
\newcommand{\blar}{\ov{\lambda}{}_R}
\newcommand{\blal}{\ov{\lambda}{}_L}
\newcommand{\blarl}{\ov{\lambda}{}_{R,L}}
\newcommand{\blalr}{\ov{\lambda}{}_{L,R}}

\newcommand{\bgamma}{\ov{\gamma}}
\newcommand{\bpsi}{\ov{\psi}{}}
\newcommand{\bphi}{\ov{\phi}{}}
\newcommand{\bxi}{\ov{\xi}{}}

\newcommand{\ff}{\mc{F}}
\newcommand{\bff}{\ov{\mc{F}}}

\newcommand{\eer}{\epsilon_R}
\newcommand{\eel}{\epsilon_L}
\newcommand{\eerl}{\epsilon_{R,L}}
\newcommand{\eelr}{\epsilon_{L,R}}
\newcommand{\beer}{\ov{\epsilon}{}_R}
\newcommand{\beel}{\ov{\epsilon}{}_L}
\newcommand{\beerl}{\ov{\epsilon}{}_{R,L}}
\newcommand{\beelr}{\ov{\epsilon}{}_{L,R}}

\newcommand{\bi}{{\bar \imath}}
\newcommand{\bj}{{\bar \jmath}}
\newcommand{\bk}{{\bar k}}
\newcommand{\bl}{{\bar l}}
\newcommand{\bmm}{{\bar m}}
\newcommand{\bp}{{\bar p}}
\newcommand{\bkk}{{\bar k}}
\newcommand{\br}{{\bar r}}

\newcommand{\nz}{{n^{(0)}}}
\newcommand{\no}{{n^{(1)}}}
\newcommand{\bnz}{{\ov{n}{}^{(0)}}}
\newcommand{\bno}{{\ov{n}{}^{(1)}}}
\newcommand{\Dz}{{D^{(0)}}}
\newcommand{\Do}{{D^{(1)}}}
\newcommand{\bDz}{{\ov{D}{}^{(0)}}}
\newcommand{\bDo}{{\ov{D}{}^{(1)}}}
\newcommand{\sigz}{{\sigma^{(0)}}}
\newcommand{\sigo}{{\sigma^{(1)}}}
\newcommand{\bsigz}{{\ov{\sigma}{}^{(0)}}}
\newcommand{\bsigo}{{\ov{\sigma}{}^{(1)}}}

\newcommand{\rrenz}{{r_\text{ren}^{(0)}}}
\newcommand{\bren}{{\beta_\text{ren}}}

\newcommand{\mbps}{m_{\text{\tiny BPS}}}
\newcommand{\W}{\mathcal{W}}
\newcommand{\hsigma}{{\hat{\sigma}}}


%%%%%%%%%%%%%%%%%%%%%%%

\begin{document}

\hyphenation{con-fi-ning}
\hyphenation{Cou-lomb}
\hyphenation{Yan-ki-e-lo-wicz}
\hyphenation{di-men-si-on-al}

%%%%%%%%%%%%%%%%%%%%%%%%%%%%%%%%


\begin{titlepage}

\begin{flushright}
FTPI-MINN-xx/xx, UMN-TH-xxxx/xx\\
September 12, 2011
\end{flushright}

\vspace{1.1cm}

\begin{center}
{  \Large \bf  
			2D -- 4D Correspondence: Towers of Kinks Versus\\[1mm]  Towers of Monopoles
			in \boldmath{${\mathcal N}=2$} Theories
			
}
\end{center}
\vspace{0.6cm}

\begin{center}

 {\large
 \bf   Pavel A.~Bolokhov$^{\,a,b}$,  Mikhail Shifman$^{\,a}$ and \bf Alexei Yung$^{\,\,a,c}$}
\end {center}

\begin{center}

%\vspace{3mm}
$^a${\it  William I. Fine Theoretical Physics Institute, University of Minnesota,
Minneapolis, MN 55455, USA}\\
$^b${\it Theoretical Physics Department, St.Petersburg State University, Ulyanovskaya~1, 
	 Peterhof, St.Petersburg, 198504, Russia}\\
$^{c}${\it Petersburg Nuclear Physics Institute, Gatchina, St. Petersburg
188300, Russia
}
\end{center}


\vspace{0.7cm}


\begin{center}
{\large\bf Abstract}
\end{center}

\hspace{0.3cm}
\vspace{2cm}

\end{titlepage}

\newpage

%%%%%%%%%%%%%%%%%%%%%%%%%%%%%%%%%%%%%%%%%%%%%%%%%%%%%%%%%%%%%%%%%%%%%%%%%%%%%%%%%%
%                                                                                %
%                          I N T R O D U C T I O N                               %
%                                                                                %
%%%%%%%%%%%%%%%%%%%%%%%%%%%%%%%%%%%%%%%%%%%%%%%%%%%%%%%%%%%%%%%%%%%%%%%%%%%%%%%%%%
\section{Introduction}
\setcounter{equation}{0}


Recently we revisited the problem of  the BPS kink spectrum in 
	the supersymmetric CP($N-1$) l model with $\mc{Z}_N$-symmetric twisted masses 
	\cite{Bolokhov:2011mp} in connection with the studies of the curves of marginal stability.
	We derived the BPS spectrum by combining three requirements: (i) at small values of the mass terms, i.e. at strong 
	coupling, the solution implied by the mirror representation \cite{MR1,MR2}, in the first order in the twisted 
	masses \cite{Shifman:2010id}; (ii) consistency in the Argyres--Douglas points, and (iii) quasiclassical limit which had been analyzed previously in \cite{Dor}. 
	Our analysis is based on the  superpotential of the Veneziano--Yankielowicz type that is
	exact in the BPS sector. This potential is presented by a multibranch (and, hence, multivalued)
	function. Therefore, a disambiguation is necessary. The combination of the above three requirements led us
	to an unambiguous prediction. A 
	a surprising finding is: in the ${\mathcal N} =2$ CP($N-1$) model with  $\mc{Z}_N$-symmetric twisted masses
	there are $N-1$ towers of BPS saturated kinks. The previous studies in the literature mention a single tower. 
	Only this single  tower is seen in the quasiclassical analysis in \cite{Dor}.
	
	Since the CP$(N -1)$ model with $\mc{Z}_N$ -symmetric twisted masses appears as a low-energy theory on
	the world sheet of non-Abelian strings \cite{Shifman:2004dr} supported in certain four-dimensional ${\mathcal N} =2$
	gauge theories with $N=N_f$, the prediction for the BPS spectrum in two dimensions can be elevated to four dimensions.
	Thus, our formula simultaneously describes confined monopoles in the Higgs phase of the four-dimensional gauge theory,
	as explained e.g. in the review paper   \cite{Shifman:2007ce}.
	Thus, we predict that at large values of the mass differences of the (s)quark fields
	(which translate into the twisted masses in 2D)  these monopoles appear in the spectrum in the form of 
	the same $N-1$ towers.
	
	In this paper we will discuss the origin and the physical meaning of the phenomenon of $N-1$ towers for 
	kinks in 2D/monopoles in 4D. To avoid bulky notation and excessive 
	technicalities we will mostly focus on the simplest nontrivial example,
	that of CP(2). Generalization to CP$(N-1)$ is conceptually straightforward. We will briefly discuss it at the end.
	
	For arbitrary $N$ the $Z_N$ symmetric twisted mass parameters are defined as
	\beq
\label{mcirc}
	m_k = m_0 \cdot e^{2 \pi i k / N}\,,\qquad k=0,1, ..., N-1\,;
\eeq
	the set of the mass parameters depends on  a single complex parameter $ m_0 $.
	In what follows we will assume $m_0$ to be real. For $N=3$ we have three mass parameters, 
	and two masses in the geometric formulation (they can be viewed as mass terms of the 
	elementary fermion excitations),
	namely
	\begin{eqnarray}
	&& m_0,\,\,\, m_1 = m_0 e^{2 \pi i  / 3},\,\,\, m_2 = m_0 e^{-2 \pi i  / 3};
	\nonumber\\[2mm]
	&&M_1 = m_1-m_0\, , \qquad M_2 = m_2 -m_0 \,.
	\end{eqnarray}
	The master formula to be used below takes the form
	\begin{eqnarray}
	M_{\rm kink} &=&
	 U_0 (m_0) + i\, \vec{N} \cdot \vec{m}\,,\nonumber\\[3mm]
	U_0 (m_0) 
	 &=& 
 -\, \frac{1}{2\pi} \left(  e^{2\pi i / 3} \,-\, 1 \right)
 \nonumber\\[3mm]
 &\times&
	\left\{ 3\, \sqrt[3] { m_0^3 \,+\, \Lambda^3 }  +
		\sum_{j=0}^2 \, m_j\, \ln \, \frac{ \sqrt[3] { m_0^3 \,+\, \Lambda^3 } \,-\, m_j } { \Lambda} \,\right\}\,,
	\end{eqnarray}
	where
	\beq
	\vec{m} = \left\{ m_0,~ ...,~ m_{N-1}\right\}\,,
	\eeq 
	and $\vec N$ is an integer-valued vector determined in \cite{Bolokhov:2011mp}. In fact, there are two such vectors (hence, two towers mentioned above),
\begin{eqnarray}
\label{ncp2}
	\vec{N}_1 &=& \left\{ \, -n \,+\, 1,~~ n,~~ 0 \,\right\},
	\nonumber\\[2mm]
	\vec{N}_2 &=& \left\{ \, -n ,~~ n,~~ 1 \,\right\},
	\label{15}
\end{eqnarray}
where $n$ is an integer parameter.

With our parameter values $ U_0 (m_0)$ is an explicit single-valued function. The multivaluedness 
resides in (\ref{15}). Indeed,
\begin{eqnarray}
\vec{m} \, \vec{N}_1  & =&
 n M_1 +m_0 = n^\prime \, M_1 + m_1\,,
\nonumber\\[2mm]
\vec{m} \, \vec{N}_2  & =&
 n\, M_1 + m_2\,,
 \nonumber\\[2mm]
n^\prime &=&
n-1\,.
\end{eqnarray}

	



\newpage

%%%%%%%%%%%%%%%%%%%%%%%%%%%%%%%%%%%%%%%%%%%%%%%%%%%%%%%%%%%%%%%%%%%%%%%%%%%%%%%%%%
%                                                                                %
%                          S E M I C L A S S I C S                               %
%                                                                                %
%%%%%%%%%%%%%%%%%%%%%%%%%%%%%%%%%%%%%%%%%%%%%%%%%%%%%%%%%%%%%%%%%%%%%%%%%%%%%%%%%%
\section{Formulation}
\setcounter{equation}{0}

       The classical expression for the central charge has two contributions \cite{ls1}:
       the Noether and the topological terms,
\beq
        \mc{Z} ~~=~~ i\, M_a\, q^a  ~~+~~ \int\, dz\, \p_z\, O \,, \qquad\qquad
	a ~=~ 1,\,...\, N-1\,.
	\label{21}
\eeq
       where $ M^a $ are the twisted masses (in the geometric formulation),
\beq
       M_a  ~~=~~ m^a ~-~ m^0\,,
\eeq
$m^a$ (a=1,2, ..., N) are the masses in the gauge formulation, and the operator $O$ consists of two parts, canonical and anomalous,
\begin{eqnarray}
O & = &
O_{\rm canon} + O_{\rm anom},\\[2mm]
O_{\rm canon } 
\label{23}
& = &
\sum_{a= 1}^{N-1} M_a D^a,\\[2mm]
O_{\rm anom} 
\label{24}
& = &
- \frac{N\, g_0^2}{4\pi}\left(
\sum_{a=1}^{N-1} M_a D^a + g_{i\bar{j}} \,\bar \psi^{\bar{j}}\,\frac{1-\gamma_5}{2}\,\psi^i
\right).
\label{25}
\end{eqnarray}
Moreover, the Noether charges $ q^a $ can be obtained from $N-1$ U(1) currents $ J_\mu^a $
defined as\,\footnote{There is a typo in the definition of these currents in \cite{ls1}.}
\begin{eqnarray}
%
       J_{RL}^a  
       &=&
        g_{i\bj}\; \bphi^\bj\, (T^a)^{i\bj}\, i\overleftrightarrow{\p}{}_{\scriptscriptstyle \!\!\!\!RL}\, \phi^i   \nonumber\\[2mm]
%
                 & +&
                  \frac{1}{2}\, g_{i\bj}\; \bpsi{}_{\scriptscriptstyle LR}^\bmm 
                         \lgr  ( T^a )_\bmm^{\ \bp}\, \delta_\bp^{\ \bj} ~+~ 
	                       \bphi{}^\br\, (T^a)_\br^{\ \bkk}\, \Gamma^{\;\bj}_{\bkk\bmm} \rgr   \psi_{\scriptscriptstyle LR}^i \nonumber \\[2mm]
%
                 & +&
                  \frac{1}{2}\, g_{i\bj}\; \bpsi{}_{\scriptscriptstyle LR}^\bj
                         \lgr  \delta^i_{\ p}\, (T^a)^p_{\ m} ~+~ 
		               \Gamma_{mk}^{\;i}\, (T^a)^k_{\ r}\, \phi^r \rgr   \psi_{\scriptscriptstyle LR}^m
\label{u1cur}
\end{eqnarray}
       in the geometric representation, and
\begin{align}
%
\notag
       J_{RL}^a  & ~~=~~ i\, \ov{n}{}_a \overleftrightarrow{\p}_{\scriptscriptstyle RL} n^a 
                   ~~-~~ |n^a|^2 \cdot i\, ( \ov{n} \overleftrightarrow{\p} n ) \\
%
                 & ~~+~~ \qquad 
                         \bxi{}_{\scriptscriptstyle LR}^a\, \xi_{\scriptscriptstyle LR}^a  ~~-~~
                         |n^a|^2 \cdot ( \bxi{}_{\scriptscriptstyle LR}\, \xi_{\scriptscriptstyle LR} )
\end{align}
       in the gauged formulation. Here
 \beq
 \left(T^a
 \right)^i_k = \delta_a^i\delta^a_k \,,\quad (\mbox{no summation over $a$!})
 \eeq
 and a similar expression for the overbarred indices.
Finally,
       $ D^a $ are the Killing potentials,
\beq
       D^a  ~~=~~ r_0\, \frac{ \bphi\,\, T^a\, \phi } 
                           {  1  ~+~  |\phi|^2  }
            ~~=~~ r_0\, \frac{ \bphi^a\, \phi^a   }
                           {  1  ~+~  |\phi|^2  }\,.
                           \label{29}
\eeq
       The generators $ T^a $ always pick up the $ a $-th component.
       In this expression, 
       \beq
       r _0=\frac{ 2 }{ g_0^2}
        \eeq 
        is a popular alternative notation for the sigma model coupling.
        
        Note that Eq. (\ref{29}) contains the bare coupling. It is clear that the one-loop correction must (and will) convert the bare coupling  into the renormalized coupling. The anomalous part $O_{\rm anom}$ is obtained at one loop.
        Therefore, in the one-loop approximation for the central charge 
        it is sufficient to treat $O_{\rm anom}$ in the lowest order. Moreover, the bifermion term 
        in $O_{\rm anom}$ plays a role only in the two-loop approximation. As a result, to calculate the central charge at one loop it is sufficient to analyze 
        the one-loop correction to $O_{\rm canon}$. The latter is determined by the tadpole graphs in Fig. XXX. As usual, the simplest way to perform the calculation is the background field method.
The part of the central charge under consideration is determined by the value of the fields at the spatial infinities. 
In the CP(2) model to be considered below there are three vacua and three possible ways
of interpolation between them. All kinks are equivalent. We will choose a  particular kink
corresponding to the following boundary conditions:
\beq
\phi^1 (z=-\infty ) = 0\,,\qquad \phi^1 (z=+\infty ) = \infty\,,\qquad  \phi^2 = 0\,.
\label{211}
\eeq
We split the field $\phi$ into two parts,
\beq
\phi = \phi_{\rm b} + \phi_{\rm qu}\,,
\label{212}
\eeq
 and expand $D^a$ in $\phi_{\rm qu}$ keeping  terms quadratic in $\phi_{\rm qu}$.
 
 
 
     
\newpage

A digression is in order regarding the one-loop central charge in the gauged formulation.
 $ D^a $ receives a one-loop contribution, due to $ r $, as is 
       most easily seen in the gauge representation of this operator,
\beq
       D^a  ~~=~~ r \cdot \ov{n}{}_a\, n^a \,,\qquad\qquad   | n_l |^2 ~=~ 1\,.
\eeq
       The renormalized operator contains the running coupling 
\beq
       r    ~~=~~ r_0  ~-~ \frac{N}{2\pi}\,\ln\, \frac{M_\text{UV}}
                                                      {   |M^a|   } \,.
\eeq
       In general, it is some typical mass scale that appears in the denominator of the logarithm, but, {\it e.g.}
       in CP(2) with $\mc{Z}_3$ twisted mass terms all masses and all mass differences have equal magnitude. 
       The UV cut-off $ M_\text{UV} $ and bare constant $ r_0 $ can be turned 
       into the strong coupling scale $ \Lambda $:
\beq
       r    ~~=~~ \frac{N}{2\pi}\, \ln\, \frac{   |M^a|   }
                                              {  \Lambda  }\,.
\eeq


       We will be looking for the semi-classical expression for the central charge in the presence
       of the soliton interpolating between vacua ({\sc \small 0}) and ({\sc \small 1})
\beq
       \phi^1(z)  \,~~=~~\, e^{|M^1| z}\,, \qquad\qquad  \phi^2(z) \,~~=~~\, \phi^3(z) \,~~=~~ \,~...~\, ~~=~~\, 0\,.
\eeq
       That this is the right kink can be seen in the gauged formulation,
\begin{align}
%
\notag
       n^0  & ~~=~~ \frac {             1              }
                          {\sqrt{ 1 ~+~ e^{2 |M^1| z} }}\,, \\[3mm]
%
\notag
       n^1  & ~~=~~ \frac {         e^{|M^1| z}        }
                          {\sqrt{ 1 ~+~ e^{2 |M^1| z} }}\,, \\[3mm]
%
\notag
       n^2  & ~~=~~ \qquad~~\, 0\,,  \\[2mm]
%	 
            & ~~~\,\vdots          \\[2mm]
%
\notag
       n^k  & ~~=~~ \qquad~~\, 0\,,  \\[2mm]
%	 
\notag
            & ~~~\,\vdots          \\[2mm]
%
\notag
       n^{N-1} & ~~=~~ \qquad~~\, 0\,.                
\end{align}

       In this background, $ D^a $ taken at the edges of the worldsheet yields just the coupling constant:
\beq
       D^a \Big|^{\scriptscriptstyle +\infty}_{\scriptscriptstyle -\infty} ~~=~~    r\,.
\eeq
       Therefore, the topological contribution to the central charge of the kink is
\beq
       \mc{Z} ~~\supset~~ \frac{N}{2\pi}\, M^1\, \ln\, \frac{   |M^a|   }
                                                            {  \Lambda  }\,.
\eeq

       As for the Noether contribution, the quantization of the ``angle'' coordinate of the kink gives 
\beq
       i\, n\, M^1\,,
\eeq
       with $ q^1 ~=~ n $ an integer number.
       As for the other $ q^k $, the kink does not have fermionic zero-modes of $ \psi^k $ with $ k = 2,\, 3,\, ...\, N-1 $.
       However, we will argue that there is a
       {\it non}-zero mode relevant to the problem of multiple towers that we consider (in fact, 
       the existence of this nonzero mode was noted by Dorey {\it et al.} \cite{Dorey:1999zk}).
       This mode describes a bound state of the kink and a fermion $ \psi^k $. 
     
     \section{Semiclassical calculation of the central charge  in CP(2)}
     \label{semclas}
     
     If the twisted masses $M_a$ satisfy the condition
     \beq
     |M_a|\gg\Lambda\,,
     \eeq
       then we find ourselves at weak coupling where the one-loop calculation of the
       central charges will be sufficient for our purposes. This calculation can be carried out in a straightforward manner for all CP$(N-1)$ models, but for the sake if simplicity we will limit ourselves to CP(2). Generalization to larger $N$ is quite obvious. 
       
In CP(2) there are two twisted mass parameters, $M_1$ and $M_2$, as shown in Fig. XXX.  Accordingly, there are two U(1) charges, see Eq. (\ref{u1cur}). The Noether charges are not renormalized; therefore we will focus on the topological part represented by the Killing potentials, which are renormalized. 

\subsection{Topological contribution}

One-loop calculations are most easily performed using the background field method.
For what follows it is important that $\phi^2_{\rm b}\equiv 0$ for the kink under consideration.
If so, all off-diagonal elements of the metric $g_{i\bar j}$ vanish, while the diagonal elements take the form
\begin{eqnarray}
&&
g_{1\bar 1}^{\rm b} \equiv g_{1\bar 1}\Big|_{\phi_{\rm b}} = \frac{2}{g_0^2}\,\frac {1}{\chi^2}\,,\nonumber\\[3mm]
&&
g_{2\bar 2}^{\rm b} \equiv 
g_{2\bar 2}\Big|_{\phi_{\rm b}} = \frac{2}{g_0^2}\,\frac {1}{\chi}\,,
\end{eqnarray}
where
\beq
\chi = 1 + \left| \phi^1_{\rm b}
\right|^2
\eeq
At the boundaries $\phi_{\rm b}^{1,2}$ take their (vacuum) coordinate-independent values;
therefore, the Lagrangian for the quantum fields can be written as
\beq
\mathcal{L}= g_{1\bar 1}^{\rm b} \left| \partial^\mu \phi^1_{\rm qu}\right|^2+
g_{2\bar 2}^{\rm b} \left| \partial^\mu \phi^2_{\rm qu}\right|^2 + ...
\label{324}
\eeq
where the ellipses stand for the terms irrelevant for our calculation.

The Killing potentials can be expanded in the same way. Under the condition $\phi^2_{\rm b}\equiv 0$ 
we arrive at
\begin{eqnarray}
D^1
&=&
 D^1\Big|_{\phi_{\rm b}} + \frac{2}{g_0^2}\, \frac{1-\left| \phi^1_{\rm b}
\right|^2}{\chi^3}\, \left|  \phi^1_{\rm qu}\right|^2 
- \frac{2}{g_0^2}\, \frac{\left| \phi^1_{\rm b}
\right|^2}{\chi^2}\, \left|  \phi^2_{\rm qu}\right|^2 + ...\,,
\nonumber\\[3mm]
D^2
&=&
0\,.
\label{325}
\end{eqnarray}
Equation (\ref{324})
implies that the Green's functions of the quantum fields are
\beq
\left\langle \phi^1_{\rm qu}, \phi^1_{\rm qu}\right\rangle = \frac{g_0^2\,\chi^2 }{2}\, \frac{i}{k^2-\left|M\right|^2}\,,\qquad
\left\langle \phi^2_{\rm qu}, \phi^2_{\rm qu}\right\rangle = \frac{g_0^2\,\chi }{2 }\, \frac{i}{k^2-\left|M\right|^2}\,.
\label{326}
\eeq
where 
\beq
\left|M\right|\equiv \left|M_1\right|\equiv \left|M_2\right|\,.
\eeq
Now, combining (\ref{325}) and (\ref{326}) to evaluate the tadpoles graphs of Fig. XXX
with $\phi^1_{\rm qu}$ and $\phi^2_{\rm qu}$ running inside
we arrive at
\begin{eqnarray}
D^1_{\rm one-loop} 
&=&
 \frac{1}{4\pi}\, \ln \frac{\left| M_{\rm uv}\right|^2}{\left|M\right|^2}
 \nonumber\\[3mm]
 &\times&
 \left( \frac{1-\left| \phi^1_{\rm b}
\right|^2}{\chi} - \frac{\left| \phi^1_{\rm b}
\right|^2}{\chi}
 \right)^{\phi_{\rm b}^{1}=0}_{\phi_{\rm b}^{1}=\infty}
  \nonumber\\[3mm]
  &=&
  \frac{1}{4\pi}\, \ln \frac{\left| M_{\rm uv}\right|^2}{\left|M\right|^2}
  \left(2 + 1
  \right).
\end{eqnarray}
where $M_{\rm uv}$ is an ultraviolet cut-off (e.g. the Pauli-Villars regulator mass). The first and second terms in the parentheses come from the $\phi^1_{\rm qu}$ and $\phi^2_{\rm qu}$ loops, respectively. In the general case of the CP$(N-1)$ model
one must replace $2+1$ by $2+ 1\times (N-2) = N$.

This information allows us to obtain the contribution of the Killing potential to the central charge at one loop, namely,
\beq
\Delta_{\rm K}{\mathcal Z} = -2M_1\left[\frac{1}{g_0^2} -
\frac{3}{4\pi} \left(\ln\left|\frac{M_{\rm uv}}{M} 
\right| +1
\right)
\right]
\label{329}
\eeq
Note that the renormalized coupling in the case at hand is \cite{Novikov:1984ac}
\beq
\frac{1}{g^2}=
\frac{1}{g_0^2} -
\frac{3}{4\pi}  \ln\left|\frac{M_{\rm uv}}{M} \right |\,.
\label{330}
\eeq
For the generic CP$(N-1)$ model the coefficient 3 in front of the logarithm 
in (\ref{330}) is
repalced by $N$.

\subsection{Contribution of the Noether charges}.

This is not the end of the story, however. We must add to $\Delta_{\rm K}{\mathcal Z}$
a part of the central charge associated with the Noether terms in (\ref{21}), which accounts for the quantziation of the
fermion zero modes as well as effects of the $\theta$ term.


\subsection{Weak-coupling expansion}
%\setcounter{equation}{0}

       We can start from the known superpotential of the Veneziano--Yankielowicz type, and the spectrum that it generates.
       It gives the  exact solution of the CP$(N-1)$ model with twisted masses in the BPS sector. 
       In our case of $\mc{Z}_N$ symmetric twisted masses this superpotential is given in \cite{Bolokhov:2011mp}.
       
       Here we narrow down to the case of CP(2), with masses that are $ \mc{Z}_3 $ symmetric.
The central charge determining the BPS spectrum is given by the difference of the values of the superpotential in  two vacua.
The general formula adjusted for CP(2) is  \cite{Bolokhov:2011mp} as follows:
\begin{eqnarray}
       \mc{Z}\Big|^{\scriptscriptstyle +\infty}_{\scriptscriptstyle -\infty} 
       &=&
       U_0(m_0)  ~~+~~ i\, n\, M_1 ~~+~~ i\, \left\{
       \begin{array}{ll}
       m_0\\[1mm]m_2
       \end{array}
       \right.
       \nonumber\\[3mm]
       &\stackrel{|m_0|\to\infty}{\longrightarrow}&
-
       \frac{3}{2\pi}\, M^1\, \Big\{ \ln\, \frac {   |M_1|   }
                                                 {  \Lambda  } -1 \Big\} +  \frac{1}{4\sqrt{3}}\; M_1\, +i\,n\, M_1
       +  i\,\left\{
       \begin{array}{ll}
       m_0\\[1mm]m_2
       \end{array}
       \right. +
       ...
        \nonumber\\[3mm]
       \label{331}
\end{eqnarray}
       where the ellipsis represents  suppressed terms dying off as inverse powers of the large mass parameter.
As was mentioned, we assume  in Eq. (\ref{331}) the parameter $ m^0 $ to be real 
and positive. (This assumption is inessential and can be easily lifted but we will not do it in this paper.)

It is not difficult to rearrange the last double-valued term presenting it as a linear combination of $m_0+m_2$ and $M_2$.
Then Equation (\ref{331}) takes the form
\beq
   \mc{Z} = 
-
       \frac{3}{2\pi}\, M^1\, \Big\{ \ln\, \frac {   |M_1|   }
                                                 {  \Lambda  } -1 \Big\} +i\,n \, M_1
                                                 -\frac{i}{4} \, M_1 \mp \frac{i}{2}\, M_2\,.
\eeq


       Theoretically it is possible to redefine $ \Lambda $ by switching on the $\theta$ term,
       which can be introduced as a phase of $\Lambda$, namely $\Lambda \to \Lambda e^{-i\theta /3}$.
       The Veneziano--Yankielowicz superpotential is defined with a ``non-perturbative'' 
       $ \Lambda_\text{np} $, which may be related to the perturbative $ \Lambda_\text{pt} $ as
       \beq
       \Lambda_\text{np}^3 =  -i \, \Lambda_\text{pt}^3\,. 
       \eeq


















%%%%%%%%%%%%%%%%%%%%%%%%%%%%%%%%%%%%%%%%%%%%%%%%%%%%%%%%%%%%%%%%%%%%%%%%%%%%%%%%%%
%                                                                                %
%                          B O U N D   S T A T E S                               %
%                                                                                %
%%%%%%%%%%%%%%%%%%%%%%%%%%%%%%%%%%%%%%%%%%%%%%%%%%%%%%%%%%%%%%%%%%%%%%%%%%%%%%%%%%
\section{Bound States}
\label{bound}

\setcounter{equation}{0}

       To find the non-zero mode, we write out the linearized Dirac equations in the background
       of the $ \phi^1 $ kink.
       For convenience, we rescale the variable $ z $ into a dimensionless variable $ s $:
\beq
       s ~~=~~ 2\, |M^1|\, z\,.
\eeq
       Then the kink takes the form
\beq
       \phi^1(s) ~~=~~ e^s\,,\qquad\qquad\text{and}\quad \phi^k(s) ~~=~~ 0 \qquad \text{for}~~ k ~>~ 1\,,
\eeq
       or
\begin{align}
%
\notag
       n^0  & ~~=~~ \frac {             1              }
                          {    \sqrt{ 1 ~+~ e^s }      }\,, \\[3mm]
%
\notag
       n^1  & ~~=~~ \frac {          e^{s/2}           }
                          {    \sqrt{ 1 ~+~ e^s }      }\,, \\[3mm]
%
\notag
       n^2  & ~~=~~ \qquad~~\, 0\,,  \\[2mm]
%	 
            & ~~~\,\vdots          \\[2mm]
%
\notag
       n^k  & ~~=~~ \qquad~~\, 0\,,  \\[2mm]
%	 
\notag
            & ~~~\,\vdots          \\[2mm]
%
\notag
       n^{N-1} & ~~=~~ \qquad~~\, 0\,.                
\end{align}

       The masses will also turn dimensionless by the same factor,
\beq
       \mu^l  ~~=~~ \frac{ m^l }
                        {2 |M^1|}\,,
	\qquad
	\text{and}
	\qquad
	\mu_G^a ~~=~~ \frac{ M^a }
                          {2 |M^1|}\,,
\eeq
       written both for geometric and gauge formulations.

       The linearized Dirac equations for the fermion $ \psi^k $ with $ k ~>~ 1 $ then look like
\begin{align}
%
\notag
       \Big\{ \p_s  ~-~ |\mu_G^1|\, f(s) \Big\}\,\, \psi_R^k   ~~+~~  i \lgr  \mu_G^1\, f(s)  ~-~  \mu_G^k \rgr\! \cdot \psi_L^k  
       & ~~=~~ \phantom{-} i\, \lambda\, \psi_L^k   \\[2mm]
%
       \Big\{ \p_s  ~-~ |\mu_G^1|\, f(s) \Big\}\,\, \psi_L^k   ~~-~~  i \lgr \ov{\mu}{}_G^1\, f(s)  ~-~ \ov{\mu}{}_G^k \rgr\! \cdot \psi_R^k 
       & ~~=~~ - i\, \ov{\lambda}\, \psi_R^k \,.
\end{align}
       Here $ f(s) $ is a real function
\beq
       f(s) ~~=~~ \frac{     e^s     }
                       { 1  ~+~  e^s }\,.
\eeq
       Eigenvalue $ \lambda $ is zero for zero-modes, or gives the energy for non-zero modes.
       If one starts from the gauged formulation, one arrives at a simpler system, which can be obtained from the
       above one by redefinition of the functions.
       That is, the conversion between the geometric and gauge formulations is precisely such as to remove the inhomogeneous term from the
       figure brackets,
\begin{align}
%
\notag
       \p_s\, \xi_R^k  ~~+~~  i \lgr \mu_G^1\, f(s) ~-~ \mu_G^k \rgr\! \cdot \xi_L^k  & ~~=~~ \phantom{-} i\, \lambda\, \xi_L^k \\[2mm]
%
       \p_s\, \xi_L^k  ~~-~~  i \lgr \ov{\mu}{}_G^1\, f(s) ~-~ \ov{\mu}{}_G^k \rgr\! \cdot \xi_R^k & ~~=~~ - i\, \ov{\lambda}\, \xi_R^k \,.
\end{align}
       
       This system does not allow normalizable zero modes.
       However, there is a normalizable non-zero mode with the energy given by the absolute value of 
\beq
       \lambda  ~~=~~  - \mu_G^k  ~+~ \alpha \mu_G^1\,.
\eeq
       The mode is
\begin{align}
%
\notag
       \xi_R^k  & ~~=~~  \lgr \frac{  e^{\alpha s}  }
                                   {   1 ~+~ e^s    }  \rgr^{ |\mu_G^1| }  \\[2mm]
%
       \xi_L^k  & ~~=~~  - i\, \frac{ \ov{\mu}{}_G^1 }
                                    {  | \mu_G^1 |   } \cdot \xi_R^k\,.
\end{align}
       It is normalizable as long as
\beq
       0  ~~\, < \,~~  \alpha  ~~\, < \,~~ 1\,,
\eeq
       and it is BPS if $ \alpha $ takes the special value (returning to the dimensionful masses)
\beq
       \alpha ~~=~~ \frac{|M^k|}
                         {|M^1|}\, \cos\, \text{Arg}\; \frac { M^k } 
                                                             { M^1 }\,.
\eeq
       In this case the energy of the mode equals
\beq
       | \lambda | ~~=~~ -  |M^k|\, \sin\, \text{Arg}\; \frac { M^k } 
                                                              { M^1 }\,.
\eeq

       That it is BPS can be seen from the expansion of the central charge
\beq
       |\, r \cdot M^1  ~+~ i\, M^k \,|  ~~=~~ r \cdot | M^1 |  ~~-~~ | M^k | \cdot \sin \text{Arg}\; \frac { M^k } 
                                                                                                            { M^1 }  
                                                                ~~+~~ ... \,,
\eeq
       in the large coupling constant $ r $.
       This is the central charge of the bound state of a fermion and the kink 
       as discovered by Dorey {\it et al.} \cite{Dorey:1999zk}, written semi-classically.


%%%%%%%%%%%%%%%%%%%%%%%%%%%%%%%%%%%%%%%%%%%%%%%%%%%%%%%%%%%%%%%%%%%%%%%%%%%%%%%%%%
%                                                                                %
%               W E A K   C O U P L I N G   E X P A N S I O N                    %
%                                                                                %
%%%%%%%%%%%%%%%%%%%%%%%%%%%%%%%%%%%%%%%%%%%%%%%%%%%%%%%%%%%%%%%%%%%%%%%%%%%%%%%%%%
%\section{Weak Coupling Expansion}
%\setcounter{equation}{0}

%       We can start from the exact known superpotential, and the spectrum that it generates.
%       Here we narrow down to the case of CP(2), with masses concord with $ Z^3 $ symmetry.

%       The spectrum we think is given by the difference of superpotentials at the two vacua 
%       plus the addition of masses \cite{Bolokhov:2011mp},
%\beq
%       \mc{Z}\Big|^{\scriptscriptstyle +\infty}_{\scriptscriptstyle -\infty} ~~=~~
%       U_0(m_0)  ~~+~~ i\, n\, M_1 ~~+~~ i\, m^k\,,
%       \qquad\qquad k~=~1,\,2\,.
%\eeq

%       In the semi-classical limit, this gives,
%\beq
%       \mc{Z} ~~=~~
%       \frac{3}{2\pi}\, M^1\, \Big\{ \ln\, \frac {   |M_1|   }
%                                                 {  \Lambda  } ~-~ 3 \Big\}
%       ~~+~~ i\, m^k 
%       ~~-~~ \frac{1}{4\sqrt{3}}\; M_1\, 
%       ~~+~~ ...
%\eeq
%       where the ellipsis represents the suppressed terms.
%       This expression is obtained in the limit of large masses, with $ m^0 $ held real and positive. 
%       Theoretically it is possible to redefine $ \Lambda $ such that the last unsuppressed term
%       turns into $ -\, i\, m^0 $.
%       That is, the Veneziano-Yankielowicz superpotential is defined with a ``non-perturbative'' 
%       $ \Lambda_\text{np} $, which is related to the perturbative one $ \Lambda_\text{pt} $ in such 
%       a way as to turn the unsuppressed term exactly into
%\beq
%       i \, (\, m^k ~-~ m^0 \,)  ~~=~~ i\, M_k\,.
%\eeq
%       This way the expansion of the exact superpotential would agree with the perturbative formula above.


%%%%%%%%%%%%%%%%%%%%%%%%%%%%%%%%%%%%%%%%%%%%%%%%%%%%%%%%%%%%%%%%%%%%%%%%%%%%%%%%%%
%                                                                                %
%               W E A K   C O U P L I N G   E X P A N S I O N                    %
%                                                                                %
%%%%%%%%%%%%%%%%%%%%%%%%%%%%%%%%%%%%%%%%%%%%%%%%%%%%%%%%%%%%%%%%%%%%%%%%%%%%%%%%%%
\section{Matching the Central Charges}
\setcounter{equation}{0}

       The following central charges need to meet the correspondence.
\vspace{0.6cm}

\begin{itemize}
\item
       The 4-dimensional central charge (at the root of the baryonic Higgs branch),
\beq
       \mc{Z}  ~~=~~ i\, \vec{n}{}_m \cdot \vec{a}{}_D  ~~+~~ i\, \vec{n}{}_e \cdot \vec{a}  
               ~~+~~ i\, m^a \cdot S^a ~~+~~ i\, m^k \; \vec{w}^k
\eeq

\item
       For magnetic charge one, and electric charge $ \vec{\alpha}{}_1 $ this gives, up to normalization,
\beq
       \mc{Z}  ~~=~~ i\, a_D(m_0)  ~~+~~ i\, ( m^1 ~-~ m^0 )\, n ~~+~~ i\, ( m^k ~-~ m^0 )\,.
\eeq

\item
       Our 2-d expression for the central charge gives
\beq
       \mc{Z}  ~~=~~ U_0(m_0)      ~~+~~ i\, ( m^1 ~-~ m^0 )\, n ~~+~~ i\, m^k\,. 
\eeq
       It could be that the four-dimensional $ \Lambda $ differs from the two-dimensional one, although
       then one of them would have to depend on the masses.

\item
       The above two-dimensional charge, when expanded, gives
\beq
       \mc{Z} ~~=~~        
       \frac{3}{2\pi}\, (m^1 ~-~ m^0)\, \bigg\{ \ln\, \frac {   | m^1 \,-\, m^0 |   }
                                                            {        \Lambda        } ~-~ 3 \bigg\}
       ~~+~~ i\, m^k 
       ~~-~~ \frac{1}{4\sqrt{3}}\; ( m^1 ~-~ m^0 )\, 
       ~~+~~ ...
\eeq

\item
       The perturbative result gives 
\beq
       \mc{Z} ~~=~~        
       \frac{3}{2\pi}\, (m^1 ~-~ m^0)\, \bigg\{ \ln\, \frac {   | m^1 \,-\, m^0 |   }
                                                            {        \Lambda        } ~-~ 3 \bigg\}
       ~~+~~ i\, ( m^k ~-~ m^0)
\eeq

\item
       The original classical expression is
\beq
       \mc{Z} ~~=~~ i\, (m^k ~-~ m^0)\, q^k  ~~+~~ (m^k ~-~ m^0) \cdot D^k \Big|^{\scriptscriptstyle +\infty}_{\scriptscriptstyle -\infty}
\eeq
\end{itemize}

\vspace{0.2cm}
       All these expressions must agree with each other

\newpage

%%%%%%%%%%%%%%%%%%%%%%%%%%%%%%%%%%%%%%%%%%%%%%%%%%%%%%%%%%%%%%%%%%%%%%%%%%%%%%%%%%
%                                                                                %
%                            C O N C L U S I O N                                 %
%                                                                                %
%%%%%%%%%%%%%%%%%%%%%%%%%%%%%%%%%%%%%%%%%%%%%%%%%%%%%%%%%%%%%%%%%%%%%%%%%%%%%%%%%%
\section{Conclusion}
\label{conclu}
\setcounter{equation}{0}

%%%%%%%%%%%%%%%%%%%%%%%%%%%%%%%%%%%%%%%%%%%%%%%%%%%%%%%%%%%%%%%%%%%%%%%%%%%%%%%%%%
%                                                                                %
%                        A C K N O W L E D G M E N T S                           %
%                                                                                %
%%%%%%%%%%%%%%%%%%%%%%%%%%%%%%%%%%%%%%%%%%%%%%%%%%%%%%%%%%%%%%%%%%%%%%%%%%%%%%%%%%
\section*{Acknowledgments}
The work of PAB and MS was supported by the DOE grant DE-FG02-94ER40823.
The work of AY was  supported
by  FTPI, University of Minnesota,
by the RFBR Grant No. 09-02-00457a
and by the Russian State Grant for
Scientific Schools RSGSS-65751.2010.2.
	
	

\begin{thebibliography}{99}




\bibitem{MR1}
  K.~Hori and C.~Vafa,
{\em Mirror symmetry,}
  arXiv:hep-th/0002222.
  %%CITATION = HEP-TH/0002222;%%
  
  \bibitem{MR2}
E.~Frenkel and A.~Losev,
  %``Mirror symmetry in two steps: A-I-B,''
  Commun.\ Math.\ Phys.\  {\bf 269}, 39 (2006)
  [arXiv:hep-th/0505131].
  %%CITATION = CMPHA,269,39;%%
  
  \bibitem{Shifman:2010id}
  M.~Shifman, A.~Yung,
  %``Non-Abelian Confinement in N=2 Supersymmetric QCD: Duality and Kinks on Confining Strings,''
  Phys.\ Rev.\  {\bf D81}, 085009 (2010).
  [arXiv:1002.0322 [hep-th]].

\bibitem{ls1}
  M.~Shifman, A.~Vainshtein, R.~Zwicky,
  %``Central charge anomalies in 2-D sigma models with twisted mass,''
  J.\ Phys.\ A {\bf A39}, 13005-13024 (2006).
  [hep-th/0602004].
  
  
\bibitem{Dorey:1999zk}
  N.~Dorey, T.~J.~Hollowood, D.~Tong,
  %``The BPS spectra of gauge theories in two-dimensions and four-dimensions,''
  JHEP {\bf 9905}, 006 (1999).
  [hep-th/9902134].

\bibitem{Bolokhov:2011mp}
P.~A.~Bolokhov, M.~Shifman, A.~Yung,
{\em BPS Spectrum of Supersymmetric {\rm CP}$(N-1)$ Theory with $Z_N$ Twisted Masses,}
    [arXiv:1104.5241 [hep-th]].
    
\bibitem{Novikov:1984ac}
  V.A.~Novikov, M.A.~Shifman, A.I.~Vainshtein, and V.I.~Zakharov,
 {\em Two-Dimensional Sigma Models: Modeling Nonperturbative Effects of Quantum Chromodynamics,}
  Phys.\ Rept.\  {\bf 116}, 103 (1984);  A.Y.~Morozov, A.M.~Perelomov, and M.A.~Shifman,
  %``Exact Gell-mann-low Function Of Supersymmetric Kahler Sigma Models,''
  Nucl.\ Phys.\  {\bf B248}, 279 (1984).
  
  \bibitem{Dor}
N.~Dorey,
%``The BPS spectra of two-dimensional
%supersymmetric gauge theories
%with  twisted mass terms,''
JHEP {\bf 9811}, 005 (1998) [hep-th/9806056].
%%CITATION = HEP-TH 9806056;%%

\bibitem{Shifman:2004dr}
  M.~Shifman, A.~Yung,
  %``NonAbelian string junctions as confined monopoles,''
  Phys.\ Rev.\  {\bf D70}, 045004 (2004).
  [hep-th/0403149].

\bibitem{Shifman:2007ce}
  M.~Shifman, A.~Yung,
{\em Supersymmetric Solitons and How They Help Us Understand Non-Abelian Gauge Theories,}
  Rev.\ Mod.\ Phys.\  {\bf 79}, 1139 (2007).
  [hep-th/0703267].

\end{thebibliography}



\end{document}
