\documentclass[epsfig,12pt]{article}
\usepackage{epsfig}
\usepackage{graphicx}
\usepackage{rotating}

%%%%%%%%%
\usepackage{latexsym}
\usepackage{amsmath}
\usepackage{amssymb}
\usepackage{relsize}
\usepackage{geometry}
\geometry{letterpaper}
\usepackage{color}
\usepackage{bm}
%\usepackage{showlabels}
%%%%%%%%%%%%

\def\beq{\begin{equation}}
\def\eeq{\end{equation}}
\def\beqn{\begin{eqnarray}}
\def\eeqn{\end{eqnarray}}
\def\Tr{{\rm Tr}}
\newcommand{\nfour}{${\cal N}=4\;$}
\newcommand{\ntwo}{${\mathcal N}=2\;$}
\newcommand{\none}{${\mathcal N}=1\,$}
\newcommand{\ntt}{${\mathcal N}=(2,2)\,$}
\newcommand{\nzt}{${\mathcal N}=(0,2)\,$}
\newcommand{\cpn}{CP$(N-1)\,$}
\newcommand{\ca}{{\mathcal A}}
\newcommand{\cell}{{\mathcal L}}
\newcommand{\cw}{{\mathcal W}}
\newcommand{\cs}{{\mathcal S}}
\newcommand{\vp}{\varphi}
\newcommand{\pt}{\partial}
\newcommand{\ve}{\varepsilon}
\newcommand{\gs}{g^{2}}
\newcommand{\zn}{$Z_N$}
\newcommand{\cd}{${\mathcal D}$}
\newcommand{\cde}{{\mathcal D}}
\newcommand{\cf}{${\mathcal F}$}
\newcommand{\cfe}{{\mathcal F}}

\newcommand{\gsim}{\lower.7ex\hbox{$
\;\stackrel{\textstyle>}{\sim}\;$}}
\newcommand{\lsim}{\lower.7ex\hbox{$
\;\stackrel{\textstyle<}{\sim}\;$}}


%%%%%%%%%%%%%
%%%%%%%%%%%
%% common definitions
\def\stackunder#1#2{\mathrel{\mathop{#2}\limits_{#1}}}
\def\beqn{\begin{eqnarray}}
\def\eeqn{\end{eqnarray}}
\def\nn{\nonumber}
\def\baselinestretch{1.1}
\def\beq{\begin{equation}}
\def\eeq{\end{equation}}
\def\ba{\beq\new\begin{array}{c}}
\def\ea{\end{array}\eeq}
\def\be{\ba}
\def\ee{\ea}
\def\stackreb#1#2{\mathrel{\mathop{#2}\limits_{#1}}}
\def\Tr{{\rm Tr}}
%\newcommand{\gsim}{\lower.7ex\hbox{$\;\stackrel{\textstyle>}{\sim}\;$}}
% \newcommand{\lsim}{\lower.7ex\hbox{$
%\;\stackrel{\textstyle<}{\sim}\;$}}
%\newcommand{\nfour}{${\mathcal N}=4$ }
%\newcommand{\ntwo}{${\mathcal N}=2$ }
\newcommand{\ntwon}{${\mathcal N}=2$}
\newcommand{\ntwot}{${\mathcal N}= \left(2,2\right) $ }
\newcommand{\ntwoo}{${\mathcal N}= \left(0,2\right) $ }
%\newcommand{\none}{${\mathcal N}=1$ }
\newcommand{\nonen}{${\mathcal N}=1$}
%\newcommand{\vp}{\varphi}
%\newcommand{\pt}{\partial}
%\newcommand{\ve}{\varepsilon}
%\newcommand{\gs}{g^{2}}
%\newcommand{\qt}{\tilde q}
%\renewcommand{\theequation}{\thesection.\arabic{equation}}

%%
\newcommand{\p}{\partial}
\newcommand{\wt}{\widetilde}
\newcommand{\ov}{\overline}
\newcommand{\mc}[1]{\mathcal{#1}}
\newcommand{\md}{\mathcal{D}}

\newcommand{\GeV}{{\rm GeV}}
\newcommand{\eV}{{\rm eV}}
\newcommand{\Heff}{{\mathcal{H}_{\rm eff}}}
\newcommand{\Leff}{{\mathcal{L}_{\rm eff}}}
\newcommand{\el}{{\rm EM}}
\newcommand{\uflavor}{\mathbf{1}_{\rm flavor}}
\newcommand{\lgr}{\left\lgroup}
\newcommand{\rgr}{\right\rgroup}

\newcommand{\Mpl}{M_{\rm Pl}}
\newcommand{\suc}{{{\rm SU}_{\rm C}(3)}}
\newcommand{\sul}{{{\rm SU}_{\rm L}(2)}}
\newcommand{\sutw}{{\rm SU}(2)}
\newcommand{\suth}{{\rm SU}(3)}
\newcommand{\ue}{{\rm U}(1)}
%%%%%%%%%%%%%%%%%%%%%%%%%%%%%%%%%%%%%%%
%  Slash character...
\def\slashed#1{\setbox0=\hbox{$#1$}             % set a box for #1
   \dimen0=\wd0                                 % and get its size
   \setbox1=\hbox{/} \dimen1=\wd1               % get size of /
   \ifdim\dimen0>\dimen1                        % #1 is bigger
      \rlap{\hbox to \dimen0{\hfil/\hfil}}      % so center / in box
      #1                                        % and print #1
   \else                                        % / is bigger
      \rlap{\hbox to \dimen1{\hfil$#1$\hfil}}   % so center #1
      /                                         % and print /
   \fi}                                        %

%%EXAMPLE:  $\slashed{E}$ or $\slashed{E}_{t}$

%%

\newcommand{\LN}{\Lambda_\text{SU($N$)}}
\newcommand{\sunu}{{\rm SU($N$) $\times$ U(1) }}
\newcommand{\sunun}{{\rm SU($N$) $\times$ U(1)}}
\def\cfl {$\text{SU($N$)}_{\rm C+F}$ }
\def\cfln {$\text{SU($N$)}_{\rm C+F}$}
\newcommand{\mUp}{m_{\rm U(1)}^{+}}
\newcommand{\mUm}{m_{\rm U(1)}^{-}}
\newcommand{\mNp}{m_\text{SU($N$)}^{+}}
\newcommand{\mNm}{m_\text{SU($N$)}^{-}}
\newcommand{\AU}{\mc{A}^{\rm U(1)}}
\newcommand{\AN}{\mc{A}^\text{SU($N$)}}
\newcommand{\aU}{a^{\rm U(1)}}
\newcommand{\aN}{a^\text{SU($N$)}}
\newcommand{\baU}{\ov{a}{}^{\rm U(1)}}
\newcommand{\baN}{\ov{a}{}^\text{SU($N$)}}
\newcommand{\lU}{\lambda^{\rm U(1)}}
\newcommand{\lN}{\lambda^\text{SU($N$)}}
%\newcommand{\Tr}{{\rm Tr\,}}
\newcommand{\bxir}{\ov{\xi}{}_R}
\newcommand{\bxil}{\ov{\xi}{}_L}
\newcommand{\xir}{\xi_R}
\newcommand{\xil}{\xi_L}
\newcommand{\bzl}{\ov{\zeta}{}_L}
\newcommand{\bzr}{\ov{\zeta}{}_R}
\newcommand{\zr}{\zeta_R}
\newcommand{\zl}{\zeta_L}
\newcommand{\nbar}{\ov{n}}

\newcommand{\CPC}{CP($N-1$)$\times$C }
\newcommand{\CPCn}{CP($N-1$)$\times$C}

\newcommand{\lar}{\lambda_R}
\newcommand{\lal}{\lambda_L}
\newcommand{\larl}{\lambda_{R,L}}
\newcommand{\lalr}{\lambda_{L,R}}
\newcommand{\bla}{\ov{\lambda}}
\newcommand{\blar}{\ov{\lambda}{}_R}
\newcommand{\blal}{\ov{\lambda}{}_L}
\newcommand{\blarl}{\ov{\lambda}{}_{R,L}}
\newcommand{\blalr}{\ov{\lambda}{}_{L,R}}

\newcommand{\bgamma}{\ov{\gamma}}
\newcommand{\bpsi}{\ov{\psi}{}}
\newcommand{\bphi}{\ov{\phi}{}}
\newcommand{\bxi}{\ov{\xi}{}}

\newcommand{\ff}{\mc{F}}
\newcommand{\bff}{\ov{\mc{F}}}

\newcommand{\eer}{\epsilon_R}
\newcommand{\eel}{\epsilon_L}
\newcommand{\eerl}{\epsilon_{R,L}}
\newcommand{\eelr}{\epsilon_{L,R}}
\newcommand{\beer}{\ov{\epsilon}{}_R}
\newcommand{\beel}{\ov{\epsilon}{}_L}
\newcommand{\beerl}{\ov{\epsilon}{}_{R,L}}
\newcommand{\beelr}{\ov{\epsilon}{}_{L,R}}

\newcommand{\bi}{{\bar \imath}}
\newcommand{\bj}{{\bar \jmath}}
\newcommand{\bk}{{\bar k}}
\newcommand{\bl}{{\bar l}}
\newcommand{\bmm}{{\bar m}}
\newcommand{\bp}{{\bar p}}
\newcommand{\bkk}{{\bar k}}
\newcommand{\br}{{\bar r}}

\newcommand{\nz}{{n^{(0)}}}
\newcommand{\no}{{n^{(1)}}}
\newcommand{\bnz}{{\ov{n}{}^{(0)}}}
\newcommand{\bno}{{\ov{n}{}^{(1)}}}
\newcommand{\Dz}{{D^{(0)}}}
\newcommand{\Do}{{D^{(1)}}}
\newcommand{\bDz}{{\ov{D}{}^{(0)}}}
\newcommand{\bDo}{{\ov{D}{}^{(1)}}}
\newcommand{\sigz}{{\sigma^{(0)}}}
\newcommand{\sigo}{{\sigma^{(1)}}}
\newcommand{\bsigz}{{\ov{\sigma}{}^{(0)}}}
\newcommand{\bsigo}{{\ov{\sigma}{}^{(1)}}}

\newcommand{\rrenz}{{r_\text{ren}^{(0)}}}
\newcommand{\bren}{{\beta_\text{ren}}}

\newcommand{\mbps}{m_{\text{\tiny BPS}}}
\newcommand{\W}{\mathcal{W}}
\newcommand{\hsigma}{{\hat{\sigma}}}


\begin{document}

	\hspace{0.8cm}Dear Nick,


\vspace{0.8cm}

	We have analysed our knowledge of the problem with CP$^{N-1}$ and 
	now we think that we agree with most of the points that you make in your paper.
	Our general understanding and current problems are listed below, 
	but at this point our tasks may essentially overlap with yours, finding the CMS.
	We find it reasonable to suggest to combine our efforts. 
	We will treat of course any your decision with respect. 
	Even if you choose not to, there may still be results which we may need
	to communicate back and forth.
	

\vspace{0.8cm}

	Here is our general understanding of the appearance of the strong coupling states in
	CP$^{N-1}$ with $ \mc{Z}_N $ masses.
	There are $ N $ Argyres-Douglas points sitting evenly on the circle.
	Take the usual tower of dyonic kinks
\beq
\label{dyons}
	\mc{D}^{(n)} ~~=~~ \mc{W}_1 ~-~ \mc{W}_0  ~~+~~  i\,n\, (m_1 \,-\, m_0)\,.
\eeq
	Pick one AD point, say the one which we usually call AD$^0$:
\beq
	m_0^{\text{AD}^0} ~~=~~ e^{i\pi/N}\,.
\eeq
	Let us fix the conventions so that the pure ``monopole'' state ($ \mc{D}^{(n)} $ with $ n = 0 $)
	becomes massless at this point.
	The branch cuts that weave the complex plane of $ m^0 $ in agreement with the logarithms 
	and $N$-th power roots in \eqref{dyons} work such that at the next clock-wise AD point $ e^{- i \pi / N } $ (that is, AD$^{(1)}$),
	it is the dyon $ \mc{D}^{(1)} $ that becomes massless, then $ \mc{D}^{(2)} $ at AD$^{(2)}$, etc.
	Going the opposite direction, counter-clockwise, the dyon $ \mc{D}^{(-1)} $ becomes massless
	at AD$^{(N-1)}$, the $ \mc{D}^{(-2)} $ at the next one, etc.
	That is, to reinstate, it is ordinary dyonic kinks which become massless at the Argyres-Douglas points
	which are $ N $ of.
	They are not any kind of bound states.
	
	However, once becoming massless and passing through into the strong coupling region (into the primary CMS), 
	these ``dyons'' become the mirror-symmetry kinks.
	There are $ N $ of them.
	They become the bound states.
	To see that one takes all $ N $ of these kinks simultaneously and pushes them outside through the one location, say
	AD$^{(0)}$ (or, in fact, through any other place on the primary CMS curve).
	Now, out of the $ N $ states, there was one that was massless at AD$^{(0)}$, and we called it the monopole.
	That state went in through AD$^{(0)}$ and went out the same way, without surpassing any monodromy lines,
	so it obviously is still massless at AD$^{(0)}$, and hence is the same monopole.
%	The one that was massless at AD$^{(0)}$ we called the monopole, and it stays the monopole and stays massless
%	(it exits the same way it went in).
	All the other ones, when exiting through AD$^{(0)}$ do not become massless.
	If you can imagine that, we dragged different dyonic states \eqref{dyons} from AD$^{(0)}$ to the same AD$^{(0)}$, 
	each along a different closed path, so that each became massless at its designated AD point ---
	and the logarithms have combined into such a monodromy as to turn those dyons into the bound states
\beq
	( \mc{ M } \!\cdot\! \mc{Q}_k )\,.
\eeq
	If, instead, we had exited the CMS through a different AD point, say AD$^{(1)}$, then we would end up
	with the bound states involving the other quarks
\beq
	( \mc{ M } \!\cdot\! \mc{Q}_{k1} )\,,\qquad\qquad \text{with}\quad
	\mc{Q}_{k1} ~~=~~ i\, (m_k \,-\, m_1)\,.
\eeq
	An example of the branch-sliced complex plane of $ m_0 $ is shown in the picture that we attach.
	There, all (hyperbolic-looking) branch cuts are related to their logarithms by subscripts $ \sigma_* - m_k $.
	There are also two branch cuts of the cubic root (for $ \sigma_k $), which connect AD points, 
	and are deformed in the picture so as to pass near the origin.

	Although a detailed analysis of the above monodromy was performed for CP$^2$ only, I am 
	sure that it works the same way for other $ N $.
	Again, the outcome is, the states that are massless at the AD points are ordinary dyons, but the states
	that sit at the strong coupling near the origin are the ``bound states''. 

\vspace{0.8cm}
	Our recent analysis has concentrated on the weak coupling expansion of the kink masses.
	To be able to see the bound states quasi-classically (that is, not relying on the BPS relation tying
	the mass to the central charge), one needs to calculate the one-loop correction to the classical mass,
	account for the anomaly, and for the eigen-value of the corresponding fermionic non-zero mode (which was
	found in your 1999 paper too). And, compare all this to the large-$m$ expansion of the exact superpotential,
	fixing all the constants.
	This part is almost complete now, but we still list it below.

	In terms of the non-zero modes, again, we re-derived the fermionic eigen-mode equation; we also derived 
	the bosonic eigen-mode equation and showed that the two non-zero modes coincide and therefore have the
	same normalizability condition as your Eq.~(35).


%%%%%%%%%%%%%%%%%%%%%%%%%%%%%%%%%%%%%%%% PAGEBREAK %%%%%%%%%%%%%%%%%%%%%%%%%%%%%%%%%%%%%%%%
\pagebreak
%%%%%%%%%%%%%%%%%%%%%%%%%%%%%%%%%%%%%%%% PAGEBREAK %%%%%%%%%%%%%%%%%%%%%%%%%%%%%%%%%%%%%%%%
	These are the questions which we think are left in this problem.

\begin{itemize}

\item
	{\it Concurrent decay modes for odd $ N $}, here in CP$^2$ for simplicity:
\beq
\label{ddec2}
	( \mc{D}^{(\nu)} \!\cdot\! \mc{Q}_2 )  ~~\longrightarrow~~    \mc{D}^{(\nu)}  ~~+~~ \mc{Q}_2
\eeq
	and
\beq
\label{ddec21}
	( \mc{D}^{(\nu)} \!\cdot\! \mc{Q}_2 )  ~~\longrightarrow~~    \mc{D}^{(\nu+1)}  ~~+~~ \mc{Q}_{21}\,,
\eeq
	where
\beq
	Q_{21} ~~=~~ i \,( m_2 \,-\, m_1 )	
\eeq
	is the ``third'' quark.
	Which one occurs first?


	We are worried by the fact that the decay curves for the above processes look disconnected. 
	They should ideally be closed curves, otherwise one can escape the decay by sneaking through the hole.
	Furthermore, the curves should not pass through the origin (even though, contradictory enough,
	zero is the solution of the CMS conditions for these curves), since the mirror theory dictates 
	that there cannot be decays in a small vicinity of the origin.

	Another point is that, to our opinion, the absence of KS relations for the process considered in our April's paper
\beq
\label{forbidden}
	( \mc{D}^{(\nu)} \!\cdot\! \mc{Q}_2 )   ~~\longrightarrow~~   ( \mc{M} \!\cdot\! \mc{Q}_2 )  ~~+~~ \nu \cdot \mc{Q}_1
\eeq
	should be reflected in the fact that the above two decays \eqref{ddec2} and \eqref{ddec21} happen
	{\it before} the \eqref{forbidden} one.


\item
	{\it Decay curves at even $ N $}, 
\beq
	( \mc{ M } \!\cdot\! \mc{Q}_k )   ~~\longrightarrow~~   \mc{M} ~~+~~ \mc{Q}_k\,,\qquad\qquad k ~=~ 2\,,\,...~ N-1\,.
\eeq
	as well as for odd $ N $ with $ k $ other than $ (N + 1)/2 $.
	Unlike the above processes, this process happens on the way from the strong coupling to the weak coupling,
	since the bound states are not observed semiclassically.

\item
	{\it Other bound states?}
	Essentially, in the strongest coupling region, there are only $ N $ states. 
	But there might be intermediate regions where bound states with more than one quark exist.
	There may be many such states.
	They could even open additional decay channels for the above single-quark bound states.

\item
	{\it Semiclassical expansion of the soliton mass at the order $ O(r^0) $}, and matching with 
	the one-loop result.
	We have done this expansion for CP$^2$ carefully, for all complex masses $ m^0 $, and were able
	to match it with the semi-classical result up to the order of $ r $ ({\it i.e.} the logarithm).
	There is still a constant term which needs to be carefully interpreted

\end{itemize}

\vspace{0.8cm}
	\hspace{0.5cm}We are looking forward to hearing your opinion,

\vspace{0.8cm}
	\hspace{0.3cm}Sincerely,

\vspace{0.7cm}
	\hspace{-0.7cm}
	Alexei Yung, Misha Shifman and Pasha Bolokhov


\end{document}
