\documentclass[epsfig,12pt]{article}
\usepackage{epsfig}
\usepackage{graphicx}
\usepackage{rotating}

%%%%%%%%%
\usepackage{latexsym}
\usepackage{amsmath}
\usepackage{amssymb}
\usepackage{relsize}
\usepackage{geometry}
\geometry{letterpaper}
\usepackage{color}
\usepackage{bm}
%\usepackage{showlabels}
%%%%%%%%%%%%

\def\beq{\begin{equation}}
\def\eeq{\end{equation}}
\def\beqn{\begin{eqnarray}}
\def\eeqn{\end{eqnarray}}
\def\Tr{{\rm Tr}}
\newcommand{\nfour}{${\cal N}=4\;$}
\newcommand{\ntwo}{${\mathcal N}=2\;$}
\newcommand{\none}{${\mathcal N}=1\,$}
\newcommand{\ntt}{${\mathcal N}=(2,2)\,$}
\newcommand{\nzt}{${\mathcal N}=(0,2)\,$}
\newcommand{\cpn}{CP$(N-1)\,$}
\newcommand{\ca}{{\mathcal A}}
\newcommand{\cell}{{\mathcal L}}
\newcommand{\cw}{{\mathcal W}}
\newcommand{\cs}{{\mathcal S}}
\newcommand{\vp}{\varphi}
\newcommand{\pt}{\partial}
\newcommand{\ve}{\varepsilon}
\newcommand{\gs}{g^{2}}
\newcommand{\zn}{$Z_N$}
\newcommand{\cd}{${\mathcal D}$}
\newcommand{\cde}{{\mathcal D}}
\newcommand{\cf}{${\mathcal F}$}
\newcommand{\cfe}{{\mathcal F}}

\newcommand{\gsim}{\lower.7ex\hbox{$
\;\stackrel{\textstyle>}{\sim}\;$}}
\newcommand{\lsim}{\lower.7ex\hbox{$
\;\stackrel{\textstyle<}{\sim}\;$}}

\renewcommand{\theequation}{\thesection.\arabic{equation}}

%%%%%%%%%%%%%
%%%%%%%%%%%
%% common definitions
\def\stackunder#1#2{\mathrel{\mathop{#2}\limits_{#1}}}
\def\beqn{\begin{eqnarray}}
\def\eeqn{\end{eqnarray}}
\def\nn{\nonumber}
\def\baselinestretch{1.1}
\def\beq{\begin{equation}}
\def\eeq{\end{equation}}
\def\ba{\beq\new\begin{array}{c}}
\def\ea{\end{array}\eeq}
\def\be{\ba}
\def\ee{\ea}
\def\stackreb#1#2{\mathrel{\mathop{#2}\limits_{#1}}}
\def\Tr{{\rm Tr}}
%\newcommand{\gsim}{\lower.7ex\hbox{$\;\stackrel{\textstyle>}{\sim}\;$}}
% \newcommand{\lsim}{\lower.7ex\hbox{$
%\;\stackrel{\textstyle<}{\sim}\;$}}
%\newcommand{\nfour}{${\mathcal N}=4$ }
%\newcommand{\ntwo}{${\mathcal N}=2$ }
\newcommand{\ntwon}{${\mathcal N}=2$}
\newcommand{\ntwot}{${\mathcal N}= \left(2,2\right) $ }
\newcommand{\ntwoo}{${\mathcal N}= \left(0,2\right) $ }
%\newcommand{\none}{${\mathcal N}=1$ }
\newcommand{\nonen}{${\mathcal N}=1$}
%\newcommand{\vp}{\varphi}
%\newcommand{\pt}{\partial}
%\newcommand{\ve}{\varepsilon}
%\newcommand{\gs}{g^{2}}
%\newcommand{\qt}{\tilde q}
\renewcommand{\theequation}{\thesection.\arabic{equation}}

%%
\newcommand{\p}{\partial}
\newcommand{\wt}{\widetilde}
\newcommand{\ov}{\overline}
\newcommand{\mc}[1]{\mathcal{#1}}
\newcommand{\md}{\mathcal{D}}

\newcommand{\GeV}{{\rm GeV}}
\newcommand{\eV}{{\rm eV}}
\newcommand{\Heff}{{\mathcal{H}_{\rm eff}}}
\newcommand{\Leff}{{\mathcal{L}_{\rm eff}}}
\newcommand{\el}{{\rm EM}}
\newcommand{\uflavor}{\mathbf{1}_{\rm flavor}}
\newcommand{\lgr}{\left\lgroup}
\newcommand{\rgr}{\right\rgroup}

\newcommand{\Mpl}{M_{\rm Pl}}
\newcommand{\suc}{{{\rm SU}_{\rm C}(3)}}
\newcommand{\sul}{{{\rm SU}_{\rm L}(2)}}
\newcommand{\sutw}{{\rm SU}(2)}
\newcommand{\suth}{{\rm SU}(3)}
\newcommand{\ue}{{\rm U}(1)}
%%%%%%%%%%%%%%%%%%%%%%%%%%%%%%%%%%%%%%%
%  Slash character...
\def\slashed#1{\setbox0=\hbox{$#1$}             % set a box for #1
   \dimen0=\wd0                                 % and get its size
   \setbox1=\hbox{/} \dimen1=\wd1               % get size of /
   \ifdim\dimen0>\dimen1                        % #1 is bigger
      \rlap{\hbox to \dimen0{\hfil/\hfil}}      % so center / in box
      #1                                        % and print #1
   \else                                        % / is bigger
      \rlap{\hbox to \dimen1{\hfil$#1$\hfil}}   % so center #1
      /                                         % and print /
   \fi}                                        %

%%EXAMPLE:  $\slashed{E}$ or $\slashed{E}_{t}$

%%

\newcommand{\LN}{\Lambda_\text{SU($N$)}}
\newcommand{\sunu}{{\rm SU($N$) $\times$ U(1) }}
\newcommand{\sunun}{{\rm SU($N$) $\times$ U(1)}}
\def\cfl {$\text{SU($N$)}_{\rm C+F}$ }
\def\cfln {$\text{SU($N$)}_{\rm C+F}$}
\newcommand{\mUp}{m_{\rm U(1)}^{+}}
\newcommand{\mUm}{m_{\rm U(1)}^{-}}
\newcommand{\mNp}{m_\text{SU($N$)}^{+}}
\newcommand{\mNm}{m_\text{SU($N$)}^{-}}
\newcommand{\AU}{\mc{A}^{\rm U(1)}}
\newcommand{\AN}{\mc{A}^\text{SU($N$)}}
\newcommand{\aU}{a^{\rm U(1)}}
\newcommand{\aN}{a^\text{SU($N$)}}
\newcommand{\baU}{\ov{a}{}^{\rm U(1)}}
\newcommand{\baN}{\ov{a}{}^\text{SU($N$)}}
\newcommand{\lU}{\lambda^{\rm U(1)}}
\newcommand{\lN}{\lambda^\text{SU($N$)}}
%\newcommand{\Tr}{{\rm Tr\,}}
\newcommand{\bxir}{\ov{\xi}{}_R}
\newcommand{\bxil}{\ov{\xi}{}_L}
\newcommand{\xir}{\xi_R}
\newcommand{\xil}{\xi_L}
\newcommand{\bzl}{\ov{\zeta}{}_L}
\newcommand{\bzr}{\ov{\zeta}{}_R}
\newcommand{\zr}{\zeta_R}
\newcommand{\zl}{\zeta_L}
\newcommand{\nbar}{\ov{n}}

\newcommand{\CPC}{CP($N-1$)$\times$C }
\newcommand{\CPCn}{CP($N-1$)$\times$C}

\newcommand{\lar}{\lambda_R}
\newcommand{\lal}{\lambda_L}
\newcommand{\larl}{\lambda_{R,L}}
\newcommand{\lalr}{\lambda_{L,R}}
\newcommand{\bla}{\ov{\lambda}}
\newcommand{\blar}{\ov{\lambda}{}_R}
\newcommand{\blal}{\ov{\lambda}{}_L}
\newcommand{\blarl}{\ov{\lambda}{}_{R,L}}
\newcommand{\blalr}{\ov{\lambda}{}_{L,R}}

\newcommand{\bgamma}{\ov{\gamma}}
\newcommand{\bpsi}{\ov{\psi}{}}
\newcommand{\bphi}{\ov{\phi}{}}
\newcommand{\bxi}{\ov{\xi}{}}

\newcommand{\ff}{\mc{F}}
\newcommand{\bff}{\ov{\mc{F}}}

\newcommand{\eer}{\epsilon_R}
\newcommand{\eel}{\epsilon_L}
\newcommand{\eerl}{\epsilon_{R,L}}
\newcommand{\eelr}{\epsilon_{L,R}}
\newcommand{\beer}{\ov{\epsilon}{}_R}
\newcommand{\beel}{\ov{\epsilon}{}_L}
\newcommand{\beerl}{\ov{\epsilon}{}_{R,L}}
\newcommand{\beelr}{\ov{\epsilon}{}_{L,R}}

\newcommand{\bi}{{\bar \imath}}
\newcommand{\bj}{{\bar \jmath}}
\newcommand{\bk}{{\bar k}}
\newcommand{\bl}{{\bar l}}
\newcommand{\bmm}{{\bar m}}
\newcommand{\bp}{{\bar p}}
\newcommand{\bkk}{{\bar k}}
\newcommand{\br}{{\bar r}}

\newcommand{\nz}{{n^{(0)}}}
\newcommand{\no}{{n^{(1)}}}
\newcommand{\bnz}{{\ov{n}{}^{(0)}}}
\newcommand{\bno}{{\ov{n}{}^{(1)}}}
\newcommand{\Dz}{{D^{(0)}}}
\newcommand{\Do}{{D^{(1)}}}
\newcommand{\bDz}{{\ov{D}{}^{(0)}}}
\newcommand{\bDo}{{\ov{D}{}^{(1)}}}
\newcommand{\sigz}{{\sigma^{(0)}}}
\newcommand{\sigo}{{\sigma^{(1)}}}
\newcommand{\bsigz}{{\ov{\sigma}{}^{(0)}}}
\newcommand{\bsigo}{{\ov{\sigma}{}^{(1)}}}

\newcommand{\rrenz}{{r_\text{ren}^{(0)}}}
\newcommand{\bren}{{\beta_\text{ren}}}

\newcommand{\mbps}{m_{\text{\tiny BPS}}}
\newcommand{\W}{\mathcal{W}}
\newcommand{\hsigma}{{\hat{\sigma}}}


%%%%%%%%%%%%%%%%%%%%%%%

\begin{document}

\hyphenation{con-fi-ning}
\hyphenation{Cou-lomb}
\hyphenation{Yan-ki-e-lo-wicz}
\hyphenation{di-men-si-on-al}

%%%%%%%%%%%%%%%%%%%%%%%%%%%%%%%%


\begin{titlepage}

\begin{flushright}
FTPI-MINN-xx/xx, UMN-TH-xxxx/xx\\
Fructidor 14/CCXIX
\end{flushright}

\vspace{1.1cm}

\begin{center}
{  \Large \bf  
			Weak Coupling Spectrum of Supersymmetric\\[3mm]
			CP\boldmath{$^{N-1}$} Theory with Twisted Mass Terms
}
\end{center}
\vspace{0.6cm}

\begin{center}

 {\large
 \bf   Pavel A.~Bolokhov$^{\,a,b}$,  Mikhail Shifman$^{\,a}$ and \bf Alexei Yung$^{\,\,a,c}$}
\end {center}

\begin{center}

%\vspace{3mm}
$^a${\it  William I. Fine Theoretical Physics Institute, University of Minnesota,
Minneapolis, MN 55455, USA}\\
$^b${\it Theoretical Physics Department, St.Petersburg State University, Ulyanovskaya~1, 
	 Peterhof, St.Petersburg, 198504, Russia}\\
$^{c}${\it Petersburg Nuclear Physics Institute, Gatchina, St. Petersburg
188300, Russia
}
\end{center}


\vspace{0.7cm}


\begin{center}
{\large\bf Abstract}
\end{center}

\hspace{0.3cm}
\vspace{2cm}

\end{titlepage}

\newpage

%%%%%%%%%%%%%%%%%%%%%%%%%%%%%%%%%%%%%%%%%%%%%%%%%%%%%%%%%%%%%%%%%%%%%%%%%%%%%%%%%%
%                                                                                %
%                          I N T R O D U C T I O N                               %
%                                                                                %
%%%%%%%%%%%%%%%%%%%%%%%%%%%%%%%%%%%%%%%%%%%%%%%%%%%%%%%%%%%%%%%%%%%%%%%%%%%%%%%%%%
\section{Introduction}
\setcounter{equation}{0}

\newpage

%%%%%%%%%%%%%%%%%%%%%%%%%%%%%%%%%%%%%%%%%%%%%%%%%%%%%%%%%%%%%%%%%%%%%%%%%%%%%%%%%%
%                                                                                %
%                          S E M I C L A S S I C S                               %
%                                                                                %
%%%%%%%%%%%%%%%%%%%%%%%%%%%%%%%%%%%%%%%%%%%%%%%%%%%%%%%%%%%%%%%%%%%%%%%%%%%%%%%%%%
\section{Semiclassics}
\setcounter{equation}{0}

       The classical expression for the central charge has two contributions \cite{ls1} ---
       the Noether and the topological terms,
\beq
        \mc{Z} ~~=~~ i\, M^a\, q^a  ~~+~~ \int\, dz\, \p_z\, ( M^a D^a )\,, \qquad\qquad
	a ~=~ 1,\,...\, N-1\,.
\eeq
       Here $ M^a $ are the geometric masses,
\beq
       M^a  ~~=~~ m^a ~-~ m^0\,.
\eeq
       In the future will probably rename them to $ m_G^a $.

       The Noether charges $ q^a $ are found from the currents $ J_\mu^a $.
       The latter are
\begin{align}
%
\notag
       J_{RL}^a  & ~~=~~ g_{i\bj}\; \bphi^\bj\, (T^a)^{i\bj}\, i\overleftrightarrow{\p}{}_{\scriptscriptstyle \!\!\!\!RL}\, \phi^i   \\
%
                 & ~~+~~ \frac{1}{2}\, g_{i\bj}\; \bpsi{}_{\scriptscriptstyle LR}^\bmm 
                         \lgr  ( T^a )_\bmm^{\ \bp}\, \delta_\bp^{\ \bj} ~+~ 
	                       \bphi{}^\br\, (T^a)_\br^{\ \bkk}\, \Gamma^{\;\bj}_{\bkk\bmm} \rgr   \psi_{\scriptscriptstyle LR}^i  \\
%
\notag
                 & ~~+~~ \frac{1}{2}\, g_{i\bj}\; \bpsi{}_{\scriptscriptstyle LR}^\bj
                         \lgr  \delta^i_{\ p}\, (T^a)^p_{\ m} ~+~ 
		               \Gamma_{mk}^{\;i}\, (T^a)^k_{\ r}\, \phi^r \rgr   \psi_{\scriptscriptstyle LR}^m
\end{align}
       in the geometric representation, and
\begin{align}
%
\notag
       J_{RL}^a  & ~~=~~ i\, \ov{n}{}_a \overleftrightarrow{\p}_{\scriptscriptstyle RL} n^a 
                   ~~-~~ |n^a|^2 \cdot i\, ( \ov{n} \overleftrightarrow{\p} n ) \\
%
                 & ~~+~~ \qquad 
                         \bxi{}_{\scriptscriptstyle LR}^a\, \xi_{\scriptscriptstyle LR}^a  ~~-~~
                         |n^a|^2 \cdot ( \bxi{}_{\scriptscriptstyle LR}\, \xi_{\scriptscriptstyle LR} )
\end{align}
       in the gauged formulation.

       $ D^a $ are the Killing potentials,
\beq
       D^a  ~~=~~ r\, \frac{ \bphi\,\, T^a\, \phi } 
                           {  1  ~+~  |\phi|^2  }
            ~~=~~ r\, \frac{ \bphi^a\, \phi^a   }
                           {  1  ~+~  |\phi|^2  }\,.
\eeq
       The generators $ T^a $ always pick out the $ a $-th component.
       In this expression, $ r ~=~ 2 / g_0^2 $ is the sigma model coupling.

       In semi-classics, $ D^a $ receives a one-loop contribution, due to $ r $, as is 
       most easily seen in the gauge representation of this operator,
\beq
       D^a  ~~=~~ r \cdot \ov{n}{}_a\, n^a \,,\qquad\qquad   | n_l |^2 ~=~ 1\,.
\eeq
       The renormalized operator contains the running coupling 
\beq
       r    ~~=~~ r_0  ~-~ \frac{N}{2\pi}\,\ln\, \frac{M_\text{UV}}
                                                      {   |M^a|   } \,.
\eeq
       In general, it is some typical mass scale that appears in the denominator of the logarithm, but, {\it e.g.}
       in CP(2) with $\mc{Z}_3$ twisted mass terms all masses and all mass differences have equal magnitude. 
       The UV cut-off $ M_\text{UV} $ and bare constant $ r_0 $ can be turned 
       into the strong coupling scale $ \Lambda $:
\beq
       r    ~~=~~ \frac{N}{2\pi}\, \ln\, \frac{   |M^a|   }
                                              {  \Lambda  }\,.
\eeq


       We will be looking for the semi-classical expression for the central charge in the presence
       of the soliton interpolating between vacua ({\sc \small 0}) and ({\sc \small 1})
\beq
       \phi^1(z)  \,~~=~~\, e^{|M^1| z}\,, \qquad\qquad  \phi^2(z) \,~~=~~\, \phi^3(z) \,~~=~~ \,~...~\, ~~=~~\, 0\,.
\eeq
       That this is the right kink can be seen in the gauged formulation,
\begin{align}
%
\notag
       n^0  & ~~=~~ \frac {             1              }
                          {\sqrt{ 1 ~+~ e^{2 |M^1| z} }}\,, \\[3mm]
%
\notag
       n^1  & ~~=~~ \frac {         e^{|M^1| z}        }
                          {\sqrt{ 1 ~+~ e^{2 |M^1| z} }}\,, \\[3mm]
%
\notag
       n^2  & ~~=~~ \qquad~~\, 0\,,  \\[2mm]
%	 
            & ~~~\,\vdots          \\[2mm]
%
\notag
       n^k  & ~~=~~ \qquad~~\, 0\,,  \\[2mm]
%	 
\notag
            & ~~~\,\vdots          \\[2mm]
%
\notag
       n^{N-1} & ~~=~~ \qquad~~\, 0\,.                
\end{align}

       In this background, $ D^a $ taken at the edges of the worldsheet yields just the coupling constant:
\beq
       D^a \Big|^{\scriptscriptstyle +\infty}_{\scriptscriptstyle -\infty} ~~=~~    r\,.
\eeq
       Therefore, the topological contribution to the central charge of the kink is
\beq
       \mc{Z} ~~\supset~~ \frac{N}{2\pi}\, M^1\, \ln\, \frac{   |M^a|   }
                                                            {  \Lambda  }\,.
\eeq

       As for the Noether contribution, the quantization of the ``angle'' coordinate of the kink gives 
\beq
       i\, n\, M^1\,,
\eeq
       with $ q^1 ~=~ n $ an integer number.
       As for the other $ q^k $, the kink does not have fermionic zero-modes of $ \psi^k $ with $ k = 2,\, 3,\, ...\, N-1 $.
       However, as we argue, and as found by Dorey {\it et al.} \cite{Dorey:1999zk}, there is a
       {\it non}-zero mode.
       That mode describes a bound state of the kink and a fermion $ \psi^k $.

%%%%%%%%%%%%%%%%%%%%%%%%%%%%%%%%%%%%%%%%%%%%%%%%%%%%%%%%%%%%%%%%%%%%%%%%%%%%%%%%%%
%                                                                                %
%                          B O U N D   S T A T E S                               %
%                                                                                %
%%%%%%%%%%%%%%%%%%%%%%%%%%%%%%%%%%%%%%%%%%%%%%%%%%%%%%%%%%%%%%%%%%%%%%%%%%%%%%%%%%
\section{Bound States}
\setcounter{equation}{0}

       To find the non-zero mode, we write out the linearized Dirac equations in the background
       of the $ \phi^1 $ kink.
       For convenience, we rescale the variable $ z $ into a dimensionless variable $ s $:
\beq
       s ~~=~~ 2\, |M^1|\, z\,.
\eeq
       Then the kink takes the form
\beq
       \phi^1(s) ~~=~~ e^s\,,\qquad\qquad\text{and}\quad \phi^k(s) ~~=~~ 0 \qquad \text{for}~~ k ~>~ 1\,,
\eeq
       or
\begin{align}
%
\notag
       n^0  & ~~=~~ \frac {             1              }
                          {    \sqrt{ 1 ~+~ e^s }      }\,, \\[3mm]
%
\notag
       n^1  & ~~=~~ \frac {          e^{s/2}           }
                          {    \sqrt{ 1 ~+~ e^s }      }\,, \\[3mm]
%
\notag
       n^2  & ~~=~~ \qquad~~\, 0\,,  \\[2mm]
%	 
            & ~~~\,\vdots          \\[2mm]
%
\notag
       n^k  & ~~=~~ \qquad~~\, 0\,,  \\[2mm]
%	 
\notag
            & ~~~\,\vdots          \\[2mm]
%
\notag
       n^{N-1} & ~~=~~ \qquad~~\, 0\,.                
\end{align}

       The masses will also turn dimensionless by the same factor,
\beq
       \mu^l  ~~=~~ \frac{ m^l }
                        {2 |M^1|}\,,
	\qquad
	\text{and}
	\qquad
	\mu_G^a ~~=~~ \frac{ M^a }
                          {2 |M^1|}\,,
\eeq
       written both for geometric and gauge formulations.

       The linearized Dirac equations for the fermion $ \psi^k $ with $ k ~>~ 1 $ then look like
\begin{align}
%
\notag
       \Big\{ \p_s  ~-~ |\mu_G^1|\, f(s) \Big\}\,\, \psi_R^k   ~~+~~  i \lgr  \mu_G^1\, f(s)  ~-~  \mu_G^k \rgr\! \cdot \psi_L^k  
       & ~~=~~ \phantom{-} i\, \lambda\, \psi_L^k   \\[2mm]
%
       \Big\{ \p_s  ~-~ |\mu_G^1|\, f(s) \Big\}\,\, \psi_L^k   ~~-~~  i \lgr \ov{\mu}{}_G^1\, f(s)  ~-~ \ov{\mu}{}_G^k \rgr\! \cdot \psi_R^k 
       & ~~=~~ - i\, \ov{\lambda}\, \psi_R^k \,.
\end{align}
       Here $ f(s) $ is a real function
\beq
       f(s) ~~=~~ \frac{     e^s     }
                       { 1  ~+~  e^s }\,.
\eeq
       Eigenvalue $ \lambda $ is zero for zero-modes, or gives the energy for non-zero modes.
       If one starts from the gauged formulation, one arrives at a simpler system, which can be obtained from the
       above one by redefinition of the functions.
       That is, the conversion between the geometric and gauge formulations is precisely such as to remove the inhomogeneous term from the
       figure brackets,
\begin{align}
%
\notag
       \p_s\, \xi_R^k  ~~+~~  i \lgr \mu_G^1\, f(s) ~-~ \mu_G^k \rgr\! \cdot \xi_L^k  & ~~=~~ \phantom{-} i\, \lambda\, \xi_L^k \\[2mm]
%
       \p_s\, \xi_L^k  ~~-~~  i \lgr \ov{\mu}{}_G^1\, f(s) ~-~ \ov{\mu}{}_G^k \rgr\! \cdot \xi_R^k & ~~=~~ - i\, \ov{\lambda}\, \xi_R^k \,.
\end{align}
       
       This system does not allow normalizable zero modes.
       However, there is a normalizable non-zero mode with the energy given by the absolute value of 
\beq
       \lambda  ~~=~~  - \mu_G^k  ~+~ \alpha \mu_G^1\,.
\eeq
       The mode is
\begin{align}
%
\notag
       \xi_R^k  & ~~=~~  \lgr \frac{  e^{\alpha s}  }
                                   {   1 ~+~ e^s    }  \rgr^{ |\mu_G^1| }  \\[2mm]
%
       \xi_L^k  & ~~=~~  - i\, \frac{ \ov{\mu}{}_G^1 }
                                    {  | \mu_G^1 |   } \cdot \xi_R^k\,.
\end{align}
       It is normalizable as long as
\beq
       0  ~~\, < \,~~  \alpha  ~~\, < \,~~ 1\,,
\eeq
       and it is BPS if $ \alpha $ takes the special value (returning to the dimensionful masses)
\beq
       \alpha ~~=~~ \frac{|M^k|}
                         {|M^1|}\, \cos\, \text{Arg}\; \frac { M^k } 
                                                             { M^1 }\,.
\eeq
       In this case the energy of the mode equals
\beq
       | \lambda | ~~=~~ -  |M^k|\, \sin\, \text{Arg}\; \frac { M^k } 
                                                              { M^1 }\,.
\eeq

       That it is BPS can be seen from the expansion of the central charge
\beq
       |\, r \cdot M^1  ~+~ i\, M^k \,|  ~~=~~ r \cdot | M^1 |  ~~-~~ | M^k | \cdot \sin \text{Arg}\; \frac { M^k } 
                                                                                                            { M^1 }  
                                                                ~~+~~ ... \,,
\eeq
       in the large coupling constant $ r $.
       This is the central charge of the bound state of a fermion and the kink 
       as discovered by Dorey {\it et al.} \cite{Dorey:1999zk}, written semi-classically.


%%%%%%%%%%%%%%%%%%%%%%%%%%%%%%%%%%%%%%%%%%%%%%%%%%%%%%%%%%%%%%%%%%%%%%%%%%%%%%%%%%
%                                                                                %
%               W E A K   C O U P L I N G   E X P A N S I O N                    %
%                                                                                %
%%%%%%%%%%%%%%%%%%%%%%%%%%%%%%%%%%%%%%%%%%%%%%%%%%%%%%%%%%%%%%%%%%%%%%%%%%%%%%%%%%
\section{Weak Coupling Expansion}
\setcounter{equation}{0}

       We can start from the exact known superpotential, and the spectrum that it generates.
       Here we narrow down to the case of CP(2), with masses concord with $ Z^3 $ symmetry.

       The spectrum we think is given by the difference of superpotentials at the two vacua 
       plus the addition of masses \cite{Bolokhov:2011mp},
\beq
       \mc{Z}\Big|^{\scriptscriptstyle +\infty}_{\scriptscriptstyle -\infty} ~~=~~
       U_0(m_0)  ~~+~~ i\, n\, M^1 ~~+~~ i\, m^k\,,
       \qquad\qquad k~=~1,\,2\,.
\eeq

       In the semi-classical limit, this gives,
\beq
       \mc{Z} ~~=~~
       \frac{3}{2\pi}\, M^1\, \Big\{ \ln\, \frac {   |M^1|   }
                                                 {  \Lambda  } ~-~ 1 \Big\}
       ~~+~~ i\, m^k 
       ~~-~~ \frac{1}{4\sqrt{3}}\; M^1\, 
       ~~+~~ ...
\eeq
       where the ellipsis represents the suppressed terms.
       This expression is obtained in the limit of large masses, with $ m^0 $ held real and positive. 
       Theoretically it is possible to redefine $ \Lambda $ such that the last unsuppressed term
       turns into $ -\, i\, m^0 $.
       That is, the Veneziano-Yankielowicz superpotential is defined with a ``non-perturbative'' 
       $ \Lambda_\text{np} $, which is related to the perturbative one $ \Lambda_\text{pt} $ in such 
       a way as to turn the unsuppressed term exactly into
\beq
       i \, (\, m^k ~-~ m^0 \,)  ~~=~~ i\, M^k\,.
\eeq
       This way the expansion of the exact superpotential would agree with the perturbative formula above.


%%%%%%%%%%%%%%%%%%%%%%%%%%%%%%%%%%%%%%%%%%%%%%%%%%%%%%%%%%%%%%%%%%%%%%%%%%%%%%%%%%
%                                                                                %
%               W E A K   C O U P L I N G   E X P A N S I O N                    %
%                                                                                %
%%%%%%%%%%%%%%%%%%%%%%%%%%%%%%%%%%%%%%%%%%%%%%%%%%%%%%%%%%%%%%%%%%%%%%%%%%%%%%%%%%
\section{Matching the Central Charges}
\setcounter{equation}{0}

       The following central charges need to meet the correspondence.
\vspace{0.6cm}

\begin{itemize}
\item
       The 4-dimensional central charge (at the root of the baryonic Higgs branch),
\beq
       \mc{Z}  ~~=~~ i\, \vec{n}{}_m \cdot \vec{a}{}_D  ~~+~~ i\, \vec{n}{}_e \cdot \vec{a}  
               ~~+~~ i\, m^a \cdot S^a ~~+~~ i\, m^k \; \vec{w}^k
\eeq

\item
       For magnetic charge one, and electric charge $ \vec{\alpha}{}_1 $ this gives, up to normalization,
\beq
       \mc{Z}  ~~=~~ i\, a_D(m_0)  ~~+~~ i\, ( m^1 ~-~ m^0 )\, n ~~+~~ i\, ( m^k ~-~ m^0 )\,.
\eeq

\item
       Our 2-d expression for the central charge gives
\beq
       \mc{Z}  ~~=~~ U_0(m_0)      ~~+~~ i\, ( m^1 ~-~ m^0 )\, n ~~+~~ i\, m^k\,. 
\eeq
       It could be that the four-dimensional $ \Lambda $ differs from the two-dimensional one, although
       then one of them would have to depend on the masses.

\item
       The above two-dimensional charge, when expanded, gives
\beq
       \mc{Z} ~~=~~        
       \frac{3}{2\pi}\, (m^1 ~-~ m^0)\, \bigg\{ \ln\, \frac {   | m^1 \,-\, m^0 |   }
                                                            {        \Lambda        } ~-~ 3 \bigg\}
       ~~+~~ i\, m^k 
       ~~-~~ \frac{1}{4\sqrt{3}}\; ( m^1 ~-~ m^0 )\, 
       ~~+~~ ...
\eeq

\item
       The perturbative result gives 
\beq
       \mc{Z} ~~=~~        
       \frac{3}{2\pi}\, (m^1 ~-~ m^0)\, \bigg\{ \ln\, \frac {   | m^1 \,-\, m^0 |   }
                                                            {        \Lambda        } ~-~ 3 \bigg\}
       ~~+~~ i\, ( m^k ~-~ m^0)
\eeq

\item
       The original classical expression is
\beq
       \mc{Z} ~~=~~ i\, (m^k ~-~ m^0)\, q^k  ~~+~~ (m^k ~-~ m^0) \cdot D^k \Big|^{\scriptscriptstyle +\infty}_{\scriptscriptstyle -\infty}
\eeq
\end{itemize}

\vspace{0.2cm}
       All these expressions must agree with each other

\newpage

%%%%%%%%%%%%%%%%%%%%%%%%%%%%%%%%%%%%%%%%%%%%%%%%%%%%%%%%%%%%%%%%%%%%%%%%%%%%%%%%%%
%                                                                                %
%                            C O N C L U S I O N                                 %
%                                                                                %
%%%%%%%%%%%%%%%%%%%%%%%%%%%%%%%%%%%%%%%%%%%%%%%%%%%%%%%%%%%%%%%%%%%%%%%%%%%%%%%%%%
\section{Conclusion}
\label{conclu}
\setcounter{equation}{0}

%%%%%%%%%%%%%%%%%%%%%%%%%%%%%%%%%%%%%%%%%%%%%%%%%%%%%%%%%%%%%%%%%%%%%%%%%%%%%%%%%%
%                                                                                %
%                        A C K N O W L E D G M E N T S                           %
%                                                                                %
%%%%%%%%%%%%%%%%%%%%%%%%%%%%%%%%%%%%%%%%%%%%%%%%%%%%%%%%%%%%%%%%%%%%%%%%%%%%%%%%%%
\section*{Acknowledgments}
The work of MS was supported in part by DOE
grant DE-FG02-94ER408. 
The work of PAB is supported by the DOE grant DE-FG02-94ER40823.
The work of AY was  supported
by  FTPI, University of Minnesota,
by RFBR Grant No. 09-02-00457a
and by Russian State Grant for
Scientific Schools RSGSS-65751.2010.2.
	
	

\begin{thebibliography}{99}

\bibitem{ls1}
  M.~Shifman, A.~Vainshtein, R.~Zwicky,
  %``Central charge anomalies in 2-D sigma models with twisted mass,''
  J.\ Phys.\ A {\bf A39}, 13005-13024 (2006).
  [hep-th/0602004].

%\cite{Dorey:1999zk}
\bibitem{Dorey:1999zk}
  N.~Dorey, T.~J.~Hollowood, D.~Tong,
  %``The BPS spectra of gauge theories in two-dimensions and four-dimensions,''
  JHEP {\bf 9905}, 006 (1999).
  [hep-th/9902134].

%\cite{Bolokhov:2011mp}
\bibitem{Bolokhov:2011mp}
  P.~A.~Bolokhov, M.~Shifman, A.~Yung,
  %``BPS Spectrum of Supersymmetric CP(N-1) Theory with Z_N Twisted Masses,''
    [arXiv:1104.5241 [hep-th]].

  

\end{thebibliography}



\end{document}
