\documentclass[epsfig,12pt]{article}
\usepackage{epsfig}
\usepackage{graphicx}
\usepackage{rotating}

%%%%%%%%%
\usepackage{latexsym}
\usepackage{amsmath}
\usepackage{amssymb}
\usepackage{relsize}
\usepackage{geometry}
\geometry{letterpaper}
\usepackage{color}
\usepackage{bm}
%\usepackage{showlabels}
%%%%%%%%%%%%

\def\beq{\begin{equation}}
\def\eeq{\end{equation}}
\def\beqn{\begin{eqnarray}}
\def\eeqn{\end{eqnarray}}
\def\Tr{{\rm Tr}}
\newcommand{\nfour}{${\cal N}=4\;$}
\newcommand{\ntwo}{${\mathcal N}=2\;$}
\newcommand{\none}{${\mathcal N}=1\,$}
\newcommand{\ntt}{${\mathcal N}=(2,2)\,$}
\newcommand{\nzt}{${\mathcal N}=(0,2)\,$}
\newcommand{\cpn}{CP$(N-1)\,$}
\newcommand{\ca}{{\mathcal A}}
\newcommand{\cell}{{\mathcal L}}
\newcommand{\cw}{{\mathcal W}}
\newcommand{\cs}{{\mathcal S}}
\newcommand{\vp}{\varphi}
\newcommand{\pt}{\partial}
\newcommand{\ve}{\varepsilon}
\newcommand{\gs}{g^{2}}
\newcommand{\zn}{$Z_N$}
\newcommand{\cd}{${\mathcal D}$}
\newcommand{\cde}{{\mathcal D}}
\newcommand{\cf}{${\mathcal F}$}
\newcommand{\cfe}{{\mathcal F}}

\newcommand{\gsim}{\lower.7ex\hbox{$
\;\stackrel{\textstyle>}{\sim}\;$}}
\newcommand{\lsim}{\lower.7ex\hbox{$
\;\stackrel{\textstyle<}{\sim}\;$}}


%%%%%%%%%%%%%
%%%%%%%%%%%
%% common definitions
\def\stackunder#1#2{\mathrel{\mathop{#2}\limits_{#1}}}
\def\beqn{\begin{eqnarray}}
\def\eeqn{\end{eqnarray}}
\def\nn{\nonumber}
\def\baselinestretch{1.1}
\def\beq{\begin{equation}}
\def\eeq{\end{equation}}
\def\ba{\beq\new\begin{array}{c}}
\def\ea{\end{array}\eeq}
\def\be{\ba}
\def\ee{\ea}
\def\stackreb#1#2{\mathrel{\mathop{#2}\limits_{#1}}}
\def\Tr{{\rm Tr}}
%\newcommand{\gsim}{\lower.7ex\hbox{$\;\stackrel{\textstyle>}{\sim}\;$}}
% \newcommand{\lsim}{\lower.7ex\hbox{$
%\;\stackrel{\textstyle<}{\sim}\;$}}
%\newcommand{\nfour}{${\mathcal N}=4$ }
%\newcommand{\ntwo}{${\mathcal N}=2$ }
\newcommand{\ntwon}{${\mathcal N}=2$}
\newcommand{\ntwot}{${\mathcal N}= \left(2,2\right) $ }
\newcommand{\ntwoo}{${\mathcal N}= \left(0,2\right) $ }
%\newcommand{\none}{${\mathcal N}=1$ }
\newcommand{\nonen}{${\mathcal N}=1$}
%\newcommand{\vp}{\varphi}
%\newcommand{\pt}{\partial}
%\newcommand{\ve}{\varepsilon}
%\newcommand{\gs}{g^{2}}
%\newcommand{\qt}{\tilde q}
%\renewcommand{\theequation}{\thesection.\arabic{equation}}

%%
\newcommand{\p}{\partial}
\newcommand{\wt}{\widetilde}
\newcommand{\ov}{\overline}
\newcommand{\mc}[1]{\mathcal{#1}}
\newcommand{\md}{\mathcal{D}}

\newcommand{\GeV}{{\rm GeV}}
\newcommand{\eV}{{\rm eV}}
\newcommand{\Heff}{{\mathcal{H}_{\rm eff}}}
\newcommand{\Leff}{{\mathcal{L}_{\rm eff}}}
\newcommand{\el}{{\rm EM}}
\newcommand{\uflavor}{\mathbf{1}_{\rm flavor}}
\newcommand{\lgr}{\left\lgroup}
\newcommand{\rgr}{\right\rgroup}

\newcommand{\Mpl}{M_{\rm Pl}}
\newcommand{\suc}{{{\rm SU}_{\rm C}(3)}}
\newcommand{\sul}{{{\rm SU}_{\rm L}(2)}}
\newcommand{\sutw}{{\rm SU}(2)}
\newcommand{\suth}{{\rm SU}(3)}
\newcommand{\ue}{{\rm U}(1)}
%%%%%%%%%%%%%%%%%%%%%%%%%%%%%%%%%%%%%%%
%  Slash character...
\def\slashed#1{\setbox0=\hbox{$#1$}             % set a box for #1
   \dimen0=\wd0                                 % and get its size
   \setbox1=\hbox{/} \dimen1=\wd1               % get size of /
   \ifdim\dimen0>\dimen1                        % #1 is bigger
      \rlap{\hbox to \dimen0{\hfil/\hfil}}      % so center / in box
      #1                                        % and print #1
   \else                                        % / is bigger
      \rlap{\hbox to \dimen1{\hfil$#1$\hfil}}   % so center #1
      /                                         % and print /
   \fi}                                        %

%%EXAMPLE:  $\slashed{E}$ or $\slashed{E}_{t}$

%%

\newcommand{\LN}{\Lambda_\text{SU($N$)}}
\newcommand{\sunu}{{\rm SU($N$) $\times$ U(1) }}
\newcommand{\sunun}{{\rm SU($N$) $\times$ U(1)}}
\def\cfl {$\text{SU($N$)}_{\rm C+F}$ }
\def\cfln {$\text{SU($N$)}_{\rm C+F}$}
\newcommand{\mUp}{m_{\rm U(1)}^{+}}
\newcommand{\mUm}{m_{\rm U(1)}^{-}}
\newcommand{\mNp}{m_\text{SU($N$)}^{+}}
\newcommand{\mNm}{m_\text{SU($N$)}^{-}}
\newcommand{\AU}{\mc{A}^{\rm U(1)}}
\newcommand{\AN}{\mc{A}^\text{SU($N$)}}
\newcommand{\aU}{a^{\rm U(1)}}
\newcommand{\aN}{a^\text{SU($N$)}}
\newcommand{\baU}{\ov{a}{}^{\rm U(1)}}
\newcommand{\baN}{\ov{a}{}^\text{SU($N$)}}
\newcommand{\lU}{\lambda^{\rm U(1)}}
\newcommand{\lN}{\lambda^\text{SU($N$)}}
%\newcommand{\Tr}{{\rm Tr\,}}
\newcommand{\bxir}{\ov{\xi}{}_R}
\newcommand{\bxil}{\ov{\xi}{}_L}
\newcommand{\xir}{\xi_R}
\newcommand{\xil}{\xi_L}
\newcommand{\bzl}{\ov{\zeta}{}_L}
\newcommand{\bzr}{\ov{\zeta}{}_R}
\newcommand{\zr}{\zeta_R}
\newcommand{\zl}{\zeta_L}
\newcommand{\nbar}{\ov{n}}

\newcommand{\CPC}{CP($N-1$)$\times$C }
\newcommand{\CPCn}{CP($N-1$)$\times$C}

\newcommand{\lar}{\lambda_R}
\newcommand{\lal}{\lambda_L}
\newcommand{\larl}{\lambda_{R,L}}
\newcommand{\lalr}{\lambda_{L,R}}
\newcommand{\bla}{\ov{\lambda}}
\newcommand{\blar}{\ov{\lambda}{}_R}
\newcommand{\blal}{\ov{\lambda}{}_L}
\newcommand{\blarl}{\ov{\lambda}{}_{R,L}}
\newcommand{\blalr}{\ov{\lambda}{}_{L,R}}

\newcommand{\bgamma}{\ov{\gamma}}
\newcommand{\bpsi}{\ov{\psi}{}}
\newcommand{\bphi}{\ov{\phi}{}}
\newcommand{\bxi}{\ov{\xi}{}}

\newcommand{\ff}{\mc{F}}
\newcommand{\bff}{\ov{\mc{F}}}

\newcommand{\eer}{\epsilon_R}
\newcommand{\eel}{\epsilon_L}
\newcommand{\eerl}{\epsilon_{R,L}}
\newcommand{\eelr}{\epsilon_{L,R}}
\newcommand{\beer}{\ov{\epsilon}{}_R}
\newcommand{\beel}{\ov{\epsilon}{}_L}
\newcommand{\beerl}{\ov{\epsilon}{}_{R,L}}
\newcommand{\beelr}{\ov{\epsilon}{}_{L,R}}

\newcommand{\bi}{{\bar \imath}}
\newcommand{\bj}{{\bar \jmath}}
\newcommand{\bk}{{\bar k}}
\newcommand{\bl}{{\bar l}}
\newcommand{\bmm}{{\bar m}}
\newcommand{\bp}{{\bar p}}
\newcommand{\bkk}{{\bar k}}
\newcommand{\br}{{\bar r}}

\newcommand{\nz}{{n^{(0)}}}
\newcommand{\no}{{n^{(1)}}}
\newcommand{\bnz}{{\ov{n}{}^{(0)}}}
\newcommand{\bno}{{\ov{n}{}^{(1)}}}
\newcommand{\Dz}{{D^{(0)}}}
\newcommand{\Do}{{D^{(1)}}}
\newcommand{\bDz}{{\ov{D}{}^{(0)}}}
\newcommand{\bDo}{{\ov{D}{}^{(1)}}}
\newcommand{\sigz}{{\sigma^{(0)}}}
\newcommand{\sigo}{{\sigma^{(1)}}}
\newcommand{\bsigz}{{\ov{\sigma}{}^{(0)}}}
\newcommand{\bsigo}{{\ov{\sigma}{}^{(1)}}}

\newcommand{\rrenz}{{r_\text{ren}^{(0)}}}
\newcommand{\bren}{{\beta_\text{ren}}}

\newcommand{\mbps}{m_{\text{\tiny BPS}}}
\newcommand{\W}{\mathcal{W}}
\newcommand{\hsigma}{{\hat{\sigma}}}


\begin{document}

	\hspace{0.8cm}Dear Nick,

\vspace{0.8cm}
	Thank you for your message of November 22 and the draft of your paper with K. Petunin, 
	and sorry for the delay with answering caused by Thanksgiving holiday.
	We need to say that we also have been working on the problem of the states 
	that we found in our previous work, and this work is close to be complete. 
	It has delayed somewhat, but now it should be finished rather quick.

\vspace{0.8cm}
	First of all, the ``extra'' states that we were discussing {\it are} the bound states 
	of dyonic kinks and elementary states (it will be easier to call the latter colloquially just
	dyons and quarks).
	They are seen semiclassically as the fermionic non-zero modes of the Dirac equation in the
	background of a kink.
	
\vspace{0.8cm}
	Another important point is that in our previous work we used a somewhat suboptimal notations	
	of the ``wandering unity'' for the ``electric'' charges,
\beq
\label{bnd}
	U_0 (m_0) ~~+~~ i\, \vec{N} \cdot \vec{m}\,,\qquad\qquad  \vec{N} ~=~ (\, 0\,,~...~ 1\,,~ ... ~0\, )\,.
\eeq
	Although internally consistent, they might bring some confusion. 
	At first one could think that $ \vec{N} $ corresponds to the fundamental representation
	of SU($N$), which seems very tricky: semi-classically all states are root-like.
	The correct resolution is that neither $ U_0 $ nor $ \vec{N} $ directly correspond to the 
	``electric'' or ``magnetic'' charges --- they are mixed in these notations.
	Instead, roughly speaking,
\beq
\label{M}
 	U_0 (m_0) ~~+~~ i\, m_0  ~~=~~ \mc{W}_1 ~-~ \mc{W}_0  ~~\equiv~~ \mc{M}
\eeq
	is the topological contribution to the central charge (the central charge of the monopole $ \mc{M} $), or what
	one can loosely call $ a_D $ or $ m_D $, while
\beq
\label{q}
	i\, (m_k ~-~ m_0) ~~\equiv~~ \mc{Q}_k
\eeq
	is the central charge of the $ k $-th quark.
	Together they give \eqref{bnd}.
	The quantities $ \mc{W}_{0,1} $ are shorthands for $ \mc{W}(\sigma_0) $ and $ \mc{W}(\sigma_1) $.
	Function $ U_0 $ is convenient since it is continuous in the physical region, and is helpful
	in, {\it e.g.} plotting the CMS pictures.
	But the above provisions need to be understood.
	Above formulas are valid in the (rather broad) vicinity of the ``zeroth'' Argyres-Douglas point,
	that is why $ m_0 $ appears to be a preferred number.
	In the vicinity of the next AD point, $ m_1 $ will be preferred, and so on.
	We believe these notations, although maybe requiring a bit extra care with regards to branch cuts,
	are more transparent in terms of the physical quantities.
	We plan to present our work in these notations, therefore, where the monopole $ \mc{M} $
	is given by
\beq
	\mc{M} ~~=~~ \mc{W}_1 ~-~ \mc{W}_0 \,
\eeq
	and the ``electric'' charges are root-like.

\vspace{0.8cm}
	We have a few remarks regarding your draft directly. 
	In most cases, we, of course, assume the $ \mc{Z}_n $-symmetric masses.

\vspace{0.8cm}
	First, {\it concerning formula (10)}. 
	As just explained $ U_0(m_0) $ does not always give the difference of the superpotential values in two vacua,
	and, in the area adjacent to the AD point it certainly does not:
	the difference vanishes at the AD point, while $ U_0(AD) $ just gives some finite number.
	The same is true regarding the relation
\beq
	\mc{W}(\sigma_k) ~~=~~ e^{2\pi i k / n} \, \mc{W}(\sigma_0).
\eeq
	This equality also does not hold everywhere, and importantly, not near the AD point.
	However, adding the $ m_0 $ mass, one can obtain
\beq
	\mc{W}(\sigma_1) ~~=~~ e^{2\pi i / n} \, \mc{W}(\sigma_0) ~~+~~ i\,m_0\,.
\eeq
	That means that the quantity which you define as $ m_{DI} $ needs to be used with caution.
	In particular, again, $ m_{D1} - m_{D0} $ does not give the topological contribution,
	but since it always comes supplemented by $ m_0 $ (as in your formulas (21) and (22)), they
	together total into a topological contribution.

\vspace{0.8cm}
	{\it Formulae (21), (22) and onwards}. 
	Just a remark that, see {\it e.g.} Eq.~\eqref{q}, the central charge of a quark always comes
	with a factor of ``i'' times its mass (as opposed to, say $ m_{DI} $).
	This is correlated with the definition of $ m_{DI} $, in which the jumps of the logarithms
	would produce the mass of the quark, and such jumps are imaginary.
	We do not know whether you meant to have this factor of ``i'' in (21) and (22) or used 
	a different normalisation.

\vspace{0.8cm}
	{\it Formula (36)}. 
	The formula is easy to understand, but the result is puzzling.
	In our (previous and current) paper we have exhaustingly explored the case of CP$^\text{2}$, 
	but not the case of larger even $ n $.
	Still, mirror theory predicts the existence of $ n $ kinks in the strong coupling region
	regardless of the parity of $ n $.
	One of them is a monopole, but the other ones just have to be the bound states of the monopole
	and quarks. 

\vspace{0.8cm}
	{\it Formulae (38), (39)}.
	Our analysis of the decay of the bound state of a dyon and quark 
\beq
	\mc{D}^{(\nu)} \!\cdot\! \mc{Q}_2 ~~\to~~ \mc{M} \!\cdot\! \mc{Q}_2  ~~+~~ \nu \cdot \mc{Q}_1
\eeq
	was made purely on the basis of the superpotential.
	Here we show the decay written in the heuristic notations, $ \mc{D}^{(\nu)} $ being the dyon
	with the ``electric'' charge $ \nu $.
	The product on the left hand side denotes the bound state of that dyon and the second quark.
	That bound state decays into a bound state of a monopole and the same quark, plus a number of
	the first quarks.
	Their central charges are as in Eqs.~\eqref{M} and \eqref{q}.
	The decay seems quite possible if the central charges of the terms on the right 
	hand side are aligned.
	As we said above, vector $ \vec{N} $ does not directly correspond to the ``electric'' charge
	of the state, and so now we change our notations into the form
\beq
	\vec{S} ~~=~~ (\, -1\,,~...~ 1\,,~ ... ~0\, )
\eeq
	and
\beq
	\vec{T} ~~=~~ (\, -1\,, ~1\,, ~0\,,~ ... ~0\, )\,.
\eeq
	Given this change in the notation for the ``electric'' and ``magnetic'' chages,
	it quite may be that the Kontsevich-Soibelman relations for the above decay will change.
	It is not clear however whether these relations will ``profit'' from this change,
	that is, will become consistent.
	On the other hand, we think that the KS relations should respect decay modes which
	are open by the condition of alignment of the central charges.

\vspace{1.0cm}
	\hspace{0.9cm}Please let us know of any questions if there are

\vspace{0.8cm}
	\hspace{0.3cm}Sincerely,

\vspace{0.7cm}
	\hspace{-0.7cm}
	Alexei Yung, Misha Shifman and Pasha Bolokhov


\end{document}
